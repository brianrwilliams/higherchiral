% ----------------------------------------------------------------
% AMS-LaTeX Paper ************************************************
% **** -----------------------------------------------------------
\documentclass[11pt]{amsart}
\usepackage{graphicx}
\usepackage{pgf,tikz,pgfplots}
\usepackage{mathrsfs}
\usepackage{mathpple}
\usepackage{tikz-cd}
\usepackage{amsmath}
\usepackage{tikz}
\usepackage{mathdots}
\usepackage{yhmath}
\usepackage{cancel}
\usepackage{color}
\usepackage{siunitx}
\usepackage{array}
\usepackage{multirow}
\usepackage{amssymb}
\usepackage{gensymb}
\usepackage{tabularx}
\usepackage{booktabs}
\usetikzlibrary{fadings}
\usetikzlibrary{patterns}
\usetikzlibrary{shadows.blur}
\usetikzlibrary{shapes}


\newcommand{\lineW}{
  \mathbf{W}_{\tikz[baseline=-0.5ex]{\draw (0,0) -- (4mm,0);
      \fill (0,0) circle (0.5mm);
      \fill (4mm,0) circle (0.5mm);}}
}

\newcommand{\bgraphG}{
  \mathbf{G}_{\tikz[baseline=-0.5ex]{
      \coordinate (A) at (0,0);
      \coordinate (B) at (4mm,0);
      \coordinate (C) at (5mm,2mm);
      \coordinate (D) at (2.5mm,4mm);
      \coordinate (E) at (-1mm,2mm);

      \fill (A) circle (0.5mm);
      \fill (B) circle (0.5mm);
      \fill (C) circle (0.5mm);
      \fill (D) circle (0.5mm);
      \fill (E) circle (0.5mm);

      \draw (A) -- (B);
      \draw (B) -- (C);
      \draw (C) -- (D);
      \draw (D) -- (E);
      \draw (E) -- (A);

      \draw (E) -- (A);
      \draw (E) -- (B);
      \draw (E) -- (C);
      \draw (E) -- (D);
  }}
}
\newcommand{\bgraphW}{
  \mathbf{W}_{\tikz[baseline=-0.5ex]{
      \coordinate (A) at (0,0);
      \coordinate (B) at (4mm,0);
      \coordinate (C) at (5mm,2mm);
      \coordinate (D) at (2.5mm,4mm);
      \coordinate (E) at (-1mm,2mm);

      \fill (A) circle (0.5mm);
      \fill (B) circle (0.5mm);
      \fill (C) circle (0.5mm);
      \fill (D) circle (0.5mm);
      \fill (E) circle (0.5mm);

      \draw (A) -- (B);
      \draw (B) -- (C);
      \draw (C) -- (D);
      \draw (D) -- (E);
      \draw (E) -- (A);

      \draw (E) -- (A);
      \draw (E) -- (B);
      \draw (E) -- (C);
      \draw (E) -- (D);
  }}
}
\newcommand{\agraphG}{
  \mathbf{G}_{\tikz[baseline=-0.5ex]{
      \coordinate (A) at (0,0);
      \coordinate (B) at (4mm,0);
      \coordinate (C) at (2mm,2mm);
      \coordinate (D) at (2mm,-2mm);

      \fill (A) circle (0.5mm);
      \fill (B) circle (0.5mm);
      \fill (C) circle (0.5mm);
      \fill (D) circle (0.5mm);

      \draw (A) -- (C);
      \draw (A) -- (D);
      \draw (C) -- (D);
      \draw (B) -- (C);
      \draw (B) -- (D);
  }}
}
\newcommand{\agraphW}{
  \mathbf{W}_{\tikz[baseline=-0.5ex]{
      \coordinate (A) at (0,0);
      \coordinate (B) at (4mm,0);
      \coordinate (C) at (2mm,2mm);
      \coordinate (D) at (2mm,-2mm);

      \fill (A) circle (0.5mm);
      \fill (B) circle (0.5mm);
      \fill (C) circle (0.5mm);
      \fill (D) circle (0.5mm);

      \draw (A) -- (C);
      \draw (A) -- (D);
      \draw (C) -- (D);
      \draw (B) -- (C);
      \draw (B) -- (D);
  }}
}
\newcommand{\triangleG}{
  \mathbf{G}_{\tikz[baseline=-0.5ex]{\node[regular polygon, regular polygon sides=3, inner sep=0pt, draw, minimum size=4mm] (triangle) {};
      \fill (triangle.corner 1) circle (0.5mm);
      \fill (triangle.corner 2) circle (0.5mm);
      \fill (triangle.corner 3) circle (0.5mm);}}
}
\newcommand{\triangleW}{
  \mathbf{W}_{\tikz[baseline=-0.5ex]{\node[regular polygon, regular polygon sides=3, inner sep=0pt, draw, minimum size=4mm] (triangle) {};
      \fill (triangle.corner 1) circle (0.5mm);
      \fill (triangle.corner 2) circle (0.5mm);
      \fill (triangle.corner 3) circle (0.5mm);}}
}
\newcommand{\trianglel}{
  l^{\tikz[baseline=-0.5ex]{\node[regular polygon, regular polygon sides=3, inner sep=0pt, draw, minimum size=2mm] (triangle) {};
      \fill (triangle.corner 1) circle (0.4mm);
      \fill (triangle.corner 2) circle (0.4mm);
      \fill (triangle.corner 3) circle (0.4mm);}}
}

\usepackage{hyperref,xcolor}% http://ctan.org/pkg/{hyperref,xcolor}
\definecolor{wine-stain}{rgb}{0.5,0,0}
\hypersetup{
  colorlinks,
  linkcolor=wine-stain,
  linktoc=all
}

\usepackage{latexsym,bm,amsmath,amssymb}

\parskip=5pt
\usetikzlibrary{arrows}
\linespread{1.2}
\textwidth15cm \oddsidemargin=1cm \evensidemargin=1cm
\setlength{\headsep}{10pt}


% ----------------------------------------------------------------
\vfuzz2pt % Don't report over-full v-boxes if over-edge is small
\hfuzz2pt % Don't report over-full h-boxes if over-edge is small
% THEOREMS -------------------------------------------------------
\newtheorem{thm}{Theorem}[section]
\newtheorem{cor}[thm]{Corollary}
\newtheorem{lem}[thm]{Lemma}
\newtheorem{prop}[thm]{Proposition}
\theoremstyle{definition}
\newtheorem{exa}[thm]{Example}
\newtheorem{defn}[thm]{Definition}
\newtheorem{exe}[thm]{Exercise}
\theoremstyle{remark}
\newtheorem{rem}[thm]{Remark}
\numberwithin{equation}{section}
% MATH -----------------------------------------------------------
\newcommand{\norm}[1]{\left\Vert#1\right\Vert}
\newcommand{\abs}[1]{\left\vert#1\right\vert}
\newcommand{\set}[1]{\left\{#1\right\}}
\newcommand{\Real}{\mathbb R}
\newcommand{\eps}{\varepsilon}
\newcommand{\To}{\longrightarrow}
\newcommand{\BX}{\mathbf{B}(X)}
\newcommand{\A}{\mathcal{A}}
\DeclareMathOperator{\pf}{Pf}
\newcommand{\R}{\mathbf{R}}
\newcommand{\CC}{\mathbb{C}}
\newcommand{\bu}{\bullet}
\newcommand{\cF}{\mathcal{F}}
\renewcommand{\AA}{\mathbb{A}}

\newcommand{\Gui}[1]{(\textcolor{red}{ZG: #1})}

\newcommand{\gui}[1]{(\textcolor{blue}{Note: #1})}
\def\BW#1{{\textcolor{purple}{{\tt {\bf BW:} #1}}}}

% ----------------------------------------------------------------

\begin{document}

\section{Feynman graph integral construction}
In this section, we will use Feynman graph integrals in holomorphic quantum field theory to construct the unit
$L_{\infty}$ chiral operations in any dimension. 
We largely use the techniques developed in \cite{wang2024feynman}.

\subsection{$d=1$: the unit chiral algebra on $\mathbb{A}^1$} \label{s:feynman1d}
We begin by reviewing the one-dimensional unit chiral algebra $\omega_{\mathbb{A}^1}$ and how Feynman graph integrals
can be used to repackage the chiral operations.
Our most novel viewpoint is the utility of Schwinger space integrals (see appendix \ref{Schwinger spaces}) to express
these operations. 
This approach will apply more generally and allow us to construct the unit chiral operations in higher dimensions.

When $d=1$, we have seen that
$$
\mathbf{J}^{\mathbb{A}^1}_{\{1<2\}}(\omega_{\mathbb{A}^1}^{\boxtimes\{1<2\}})=\left\{f(z_{1},z_{2})dz_{1}\boxtimes dz_{2} \; | \; f\in k\left[z_{1},z_{2},\frac{1}{z_{1}-z_{2}}\right]\right\}.
$$

\begin{defn}
    We define the 2-operation $\mu_{\{1<2\}}$ by the following formula:
    $$
    \mu_{\{1<2\}}(f(z_{1},z_{2})dz_{1}\boxtimes dz_{2})=\frac{1}{2\pi i}e^{-(\lambda_{1}+\lambda_{2})w}\left. \left(\oint_{z_{1}=z_2}f(z_{1},z_{2})e^{\lambda_{1}z_{1}+\lambda_{2}z_{2}}dz_{1}\boxtimes dz_{2}\right)\right|_{z_{2}=w},
    $$
    where $\oint_{z_{1}=z_{2}}$ is the contour integral of variable $z_{1}$ over a simple loop around $z_{2}$.
  \end{defn}


  We unpack the definition to obtain a more familiar expression.
  Suppose that
$$\alpha=\frac{g(z_{1},z_{2})}{(z_{1}-z_{2})^{n}}dz_{1}\boxtimes dz_{2}\in \mathbf{J}^{\mathbb{A}^1}_{\{1<2\}}(\omega_{\mathbb{A}^1}^{\boxtimes\{1<2\}}),
$$
where $g(z_{1},z_{2})\in k[z_{1},z_{2}]$.
Expanding, we find that
\begin{align*}
 \mu_{\{1<2\}}(\alpha)
 &=\frac{1}{2\pi i}e^{-(\lambda_{1}+\lambda_{2})w}\left. \left(\oint_{z_{1}=z_2}\frac{g(z_{1},z_{2})}{(z_{1}-z_{2})^{n}}e^{\lambda_{1}z_{1}+\lambda_{2}z_{2}}dz_{1}\boxtimes dz_{2}\right)\right|_{z_{2}=w}\\
 &= e^{-(\lambda_{1}+\lambda_{2})w}\frac{1}{(n-1)!}\left.\left(\partial_{z_{1}}^{n-1}\left(g(z_{1},w)e^{\lambda_{1}z_{1}+\lambda_{2}w}\right)\right)\right|_{z_{1}=w}dw
 \\
 &=\frac{1}{(n-1)!}
\left.\left((\partial_{z_{1}}+\lambda_{1})^{n-1}g(z_{1},w)\right)\right|_{z_{1}=w}dw,
\end{align*}
so $\mu_{\{1<2\}}(\alpha)\in \omega_{\mathbb{A}^{1}}[\mathbb{A}^{1}]\otimes_{k[\lambda_{\bullet}]}k[\lambda_{1},\lambda_{2}].$
For example, when $g = 1$ this reduces to
\begin{equation}
\mu_{1<2}\left(\frac{d z_{1} \boxtimes d z_{2}}{(z_{1}-z_{2})^{n}}\right) = \frac{\lambda_{1}^{n-1}}{(n-1)!} \otimes d w
\end{equation}

The following proposition is well-known 
%can be found in \cite{??} 
and summarizes the chiral operation in the usual, one-dimensional,
context.
\begin{prop}[\cite{??}]
    $\mu_{\{1<2\}}$ defines a chiral operation, i.e., 
    $$    \mu_{\{1<2\}}\in {Hom}^{ch}(\omega_{\mathbb{A}^{1}}^{\boxtimes \{1<2\}},\omega_{\mathbb{A}^{1}}).
    $$
    and defines a chiral algebra structure on $\omega_{\mathbb{A}^1}$.
\end{prop}
%\begin{proof}
%We prove that $\mu_{\{1<2\}}$ is a $\mathcal{D}_{(\mathbb{A}^{1})^{\{1<2\}}}[(\mathbb{A}^{1})^{\{1<2\}}]$-module morphism.
%
%
%Recall that the $D$-module structure is determined by the rules
%$$
%\alpha \cdot z_{i}=z_{i}\alpha,\quad
%\alpha\cdot \partial_{z_{i}}=-\partial_{z_{i}}\left(\frac{g(z_{1},z_{2})}{(z_{1}-z_{2})^{n}}\right)dz_{1}\boxtimes dz_{2} .
%$$
%Thus, we can directly compute:
%\begin{align*}
%\mu_{\{1<2\}}(\alpha \cdot z_{i})&=\frac{1}{2\pi i}e^{-(\lambda_{1}+\lambda_{2})w}\left. \left(\oint_{z_{1}=z_2}\frac{z_{i}g(z_{1},z_{2})}{(z_{1}-z_{2})^{n}}e^{\lambda_{1}z_{1}+\lambda_{2}z_{2}}dz_{1}\boxtimes dz_{2}\right)\right|_{z_{2}=w}\\
%&=\frac{1}{2\pi i}e^{-(\lambda_{1}+\lambda_{2})w}\left. \left(\oint_{z_{1}=z_2}\frac{g(z_{1},z_{2})}{(z_{1}-z_{2})^{n}}\partial_{\lambda_{i}}\left(e^{\lambda_{1}z_{1}+\lambda_{2}z_{2}}\right)dz_{1}\boxtimes dz_{2}\right)\right|_{z_{2}=w}\\
%&=\frac{1}{2\pi i}(w+\partial_{\lambda_{i}})\left(e^{-(\lambda_{1}+\lambda_{2})w}\left. \left(\oint_{z_{1}=z_2}\frac{g(z_{1},z_{2})}{(z_{1}-z_{2})^{n}}e^{\lambda_{1}z_{1}+\lambda_{2}z_{2}}dz_{1}\boxtimes dz_{2}\right)\right|_{z_{2}=w}\right)\\
%&=\mu_{\{1<2\}}(\alpha )\cdot z_{i},
%\end{align*}
%\begin{align*}
%\mu_{\{1<2\}}(\alpha \cdot \partial_{z_{i}})&=-\frac{1}{2\pi i}e^{-(\lambda_{1}+\lambda_{2})w}\left. \left(\oint_{z_{1}=z_2}\partial_{z_{i}}\left(\frac{g(z_{1},z_{2})}{(z_{1}-z_{2})^{n}}\right)e^{\lambda_{1}z_{1}+\lambda_{2}z_{2}}dz_{1}\boxtimes dz_{2}\right)\right|_{z_{2}=w}\\
%&=-\frac{1}{2\pi i}e^{-(\lambda_{1}+\lambda_{2})w}\left. \left(\oint_{z_{1}=z_2}\partial_{z_{i}}\left(\frac{g(z_{1},z_{2})}{(z_{1}-z_{2})^{n}}e^{\lambda_{1}z_{1}+\lambda_{2}z_{2}}\right)dz_{1}\boxtimes dz_{2}\right)\right|_{z_{2}=w}\\
%&+\frac{1}{2\pi i}e^{-(\lambda_{1}+\lambda_{2})w}\left. \left(\oint_{z_{1}=z_2}\frac{g(z_{1},z_{2})}{(z_{1}-z_{2})^{n}}\partial_{z_{i}}\left(e^{\lambda_{1}z_{1}+\lambda_{2}z_{2}}\right)dz_{1}\boxtimes dz_{2}\right)\right|_{z_{2}=w}\\
%&=0+\frac{\lambda_{i}}{2\pi i}e^{-(\lambda_{1}+\lambda_{2})w}\left. \left(\oint_{z_{1}=z_2}\frac{g(z_{1},z_{2})}{(z_{1}-z_{2})^{n}}e^{\lambda_{1}z_{1}+\lambda_{2}z_{2}}dz_{1}\boxtimes dz_{2}\right)\right|_{z_{2}=w}\\
%&=\mu_{\{1<2\}}(\alpha )\cdot \partial_{z_{i}}.
%\end{align*}
%\end{proof}
%
%The special feature in dimension one is that the unit chiral algebra has trivial higher operations.
%
%\begin{prop}
%Let $\mu_{\vec{I}}=0$ for $|\vec{I}|\geq3$. Then $\{\mu_{\vec{I}}\}$ defines an $L_{\infty}$-chiral algebra structure on~$\omega_{\mathbb{A}^{1}}$.
%\end{prop}
%\begin{proof}
%    We first verify $\mu_{\{1<2\}}=-\mu_{\{2<1\}}\circ \tau_{\sigma_{12}}$.
%    
%    Let 
%$$\alpha=\frac{g(z_{1},z_{2})}{(z_{1}-z_{2})^{n}}dz_{1}\boxtimes dz_{2}\in \mathbf{J}^{\mathbb{A}^1}_{\{1<2\}}(\omega_{\mathbb{A}^1}^{\boxtimes\{1<2\}}),
%$$
%where $g(z_{1},z_{2})\in k[z_{1},z_{2}]$. 
%
%We have
%\begin{align*}
%    \mu_{\{1<2\}}(\alpha)
%    &=
%    \mu_{\{1<2\}}(\frac{g(z_{1},z_{2})}{(z_{1}-z_{2})^{n}}dz_{1}\boxtimes dz_{2})\\
%    &=
%    \mu_{\{1<2\}}(\frac{1}{(z_{1}-z_{2})^{n}}dz_{1}\boxtimes dz_{2})\cdot g(z_{1},z_{2})\\
%    &=
%    \frac{1}{2\pi i}\left(e^{-(\lambda_{1}+\lambda_{2})w}\left. \left(\oint_{z_{1}=z_2}\frac{1}{(z_{1}-z_{2})^{n}}e^{\lambda_{1}z_{1}+\lambda_{2}z_{2}}dz_{1}\boxtimes dz_{2}\right)\right|_{z_{2}=w}\right)\cdot g(z_{1},z_{2})\\
%    &=
%     \frac{1}{2\pi i}\left(\oint_{z_{1}=w}\frac{1}{(z_{1}-w)^{n}}e^{\lambda_{1}(z_{1}-w)}dz_{1}\boxtimes dw\right)\cdot g(z_{1},z_{2}).
%\end{align*}
%
%Let $z_{1}=2w-\tilde{z}_2$, we get
%\begin{align*}
%    \mu_{\{1<2\}}(\alpha)
%    &=
%    -\frac{1}{2\pi i}\left(\oint_{\tilde{z}_{2}=w}\frac{1}{(w-\tilde{z}_{2})^{n}}e^{\lambda_{2}(\tilde{z}_{2}-w)}e^{\lambda_{\bullet}(w-\tilde{z}_{2})}d\tilde{z}_{2}\boxtimes dw\right)\cdot g(z_{1},z_{2})\\
%    &=
%     -\frac{1}{2\pi i}\left(\oint_{\tilde{z}_{2}=w}\frac{1}{(w-\tilde{z}_{2})^{n}}e^{\lambda_{2}(\tilde{z}_{2}-w)}\left(\sum_{k=0}^{n-1}\frac{1}{k!}\lambda_{\bullet}^{k}(w-\tilde{z}_{2})^{k} \right)d\tilde{z}_{2}\boxtimes dw\right)\cdot g(z_{1},z_{2})\\
%     &=
%      -\frac{1}{2\pi i}\left(\oint_{\tilde{z}_{2}=w}\frac{1}{(w-\tilde{z}_{2})^{n}}e^{\lambda_{2}(\tilde{z}_{2}-w)}d\tilde{z}_{2}\boxtimes dw\right)\cdot g(z_{1},z_{2})\\
%      &-
%     \frac{1}{2\pi i}\left(\oint_{\tilde{z}_{2}=w}\frac{1}{(w-\tilde{z}_{2})^{n}}e^{\lambda_{2}(\tilde{z}_{2}-w)}\left(\sum_{k=1}^{n-1}\frac{1}{k!}\lambda_{\bullet}^{k}(w-\tilde{z}_{2})^{k} \right)d\tilde{z}_{2}\boxtimes dw\right)\cdot g(z_{1},z_{2}),
%\end{align*}
%where $\lambda_{\bullet}=\lambda_{1}+\lambda_{2}$.
%
%The first term equals
%$$
%-\mu_{\{2<1\}}(\frac{1}{(z_{1}-z_{2})^{n}}dz_{1}\boxtimes dz_{2})\cdot g(z_{1},z_{2})=-\mu_{\{2<1\}}\circ\tau_{\sigma_{12}}(\alpha),
%$$
%and the second term equals
%\begin{align*}
%    &-\frac{1}{2\pi i}\left(\oint_{\tilde{z}_{2}=0}\left(\sum_{k=1}^{n-1}\frac{1}{k!}(-\tilde{z}_{2})^{k-n} \lambda_{\bullet}^{k}e^{\lambda_{2}(\tilde{z}_{2})}\right)d\tilde{z}_{2}\boxtimes dw\right)\cdot g(z_{1},z_{2})\\
%     &=
%     \frac{\lambda_{\bullet}dw}{2\pi i}\left(\oint_{\tilde{z}_{2}=0}\left(\sum_{k=1}^{n-1}\frac{1}{k!}(-\tilde{z}_{2})^{k-n} \lambda_{\bullet}^{k-1}e^{\lambda_{2}(\tilde{z}_{2})}\right)d\tilde{z}_{2}\right)\cdot g(z_{1},z_{2}).
%\end{align*}
%
%We note $\lambda_{\bullet}dw=0$ in $\omega_{\mathbb{A}^{1}}[\mathbb{A}^{1}]\otimes_{k[\lambda_{\bullet}]}k[\lambda_{1},\lambda_{2}]$, the second term equals zero. So we proved that the two-ary operation is anti-symmetric $\mu_{\{1<2\}}=-\mu_{\{2<1\}}\circ \tau_{\sigma_{12}}.$
%
%Now we prove the $L_{\infty}$ relations.
%
%Let $\beta\in \mathbf{J}^{\mathbb{A}^1}_{\{1<2<3\}}(\omega_{\mathbb{A}^1}^{\boxtimes\{1<2<3\}})$. Consider the $3$-chain
%\begin{align*} 
%M_{w}=&\{(z_{1},z_{2},z_{3})\in (\mathbb{A}^1)^{1<2<3}|z_{1}+z_{2}+z_{3}=w,|z_{i}-z_{j}|\geq 1,\;i\neq j;\\
%&\sum_{k=1}^{3}|z_{k}-w|^{2}=100\}.
%\end{align*}
%Since $\beta e^{\sum_{k=1}^{3}z_{k}\lambda_{k}}dz_{1}\boxtimes dz_{2}\boxtimes dz_{3}$ is a closed differential form on $M_{w}$, we can use Stokes theorem to get the following result:
%\begin{equation}
%\begin{aligned}
%\label{stokes thm1}
%    0 &= 
%    \frac{1}{(2\pi i)^{2}}e^{-\lambda_{\blacksquare }w}\int_{M_{w}}d\left(\beta e^{\sum_{k=1}^{3}z_{k}\lambda_{k}}dz_{1}\boxtimes dz_{2}\boxtimes dz_{3}\right)\\
%    &=
%     \frac{1}{(2\pi i)^{2}}e^{-\lambda_{\blacksquare }w}\int_{\partial M_{w}}\beta e^{\sum_{k=1}^{3}z_{k}\lambda_{k}}dz_{1}\boxtimes dz_{2}\boxtimes dz_{3},
%\end{aligned}
%\end{equation}
%where $\lambda_{\blacksquare}=\sum_{k=1}^3\lambda_{k}$.
%The boundary of the $3$-chain $M_{w}$ ??
%The right hand side of $(\ref{stokes thm1})$ is
%\begin{align*}
%     &
%     \frac{1}{(2\pi i)^{2}}e^{-\lambda_{\blacksquare }w}\int_{\partial M_{w}}\beta e^{\sum_{k=1}^{3}z_{k}\lambda_{k}}dz_{1}\boxtimes dz_{2}\boxtimes dz_{3}\\
%     &=
%     \frac{1}{(2\pi i)^{2}}e^{-\lambda_{\blacksquare }w}\left.\left(\oint_{z_{2}=z_{3}}\oint_{z_{1}=z_{2}}\beta e^{\sum_{k=1}^{3}z_{k}\lambda_{k}}dz_{1}\boxtimes dz_{2}\boxtimes dz_{3}\right)\right|_{z_{3}=w}\\
%     &-
%     \frac{1}{(2\pi i)^{2}}e^{-\lambda_{\blacksquare }w}\left.\left(\oint_{z_{2}=z_{3}}\oint_{z_{1}=z_{3}}\beta e^{\sum_{k=1}^{3}z_{k}\lambda_{k}}dz_{1}\boxtimes dz_{2}\boxtimes dz_{3}\right)\right|_{z_{3}=w}\\
%     &+
%     \frac{1}{(2\pi i)^{2}}e^{-\lambda_{\blacksquare }w}\left.\left(\oint_{z_{1}=z_{2}}\oint_{z_{2}=z_{3}}\beta e^{\sum_{k=1}^{3}z_{k}\lambda_{k}}dz_{1}\boxtimes dz_{2}\boxtimes dz_{3}\right)\right|_{z_{3}=w}\\
%     &=
%     \mu_{\{\bullet<3\}}\circ \mu_{\{1<2\}\subset\{1<2<3\}}(\beta)\\
%     &-
%     \mu_{\{\bullet<2\}}\circ \mu_{\{1<3\}\subset\{1<2<3\}}(\beta)\\
%     &+
%     \mu_{\{\bullet<1\}}\circ \mu_{\{2<3\}\subset\{1<2<3\}}(\beta)\\
%     &=  
%     \sum_{\vec{I'}\subset\{1<2<3\}}\mathrm{sgn}(\sigma_{\vec{I'}\subset\vec{I}})(-1)^{|I'|(|I|-|I'|)}\mu_{\{\bullet\}\cup\vec{I}-\vec{I'}}\circ \mu_{\vec{I'}\subset\vec{I}}(\beta).
%\end{align*}
%
%We verified the $L_{\infty }$ relations.
%\end{proof}

In the construction of chiral algebra structure on $\omega_{\mathbb{A}^{1}}$, we used the concept of residue. Now we
introduce another method to define residues, which we will use to construct chiral operations in higher dimensions.
We begin with the following presentation of the Cauchy kernel.
\begin{lem}
    $$
    \frac{1}{z}=-\int_{0}^{+\infty}e^{-zy}dy,
    $$
    where $y=\frac{\bar{z}}{t}$, and $dy$ is viewed as the differential form $$dy=\frac{1}{t} d\bar{z}-\frac{\bar{z}}{t^2} d t.$$
    The integration is over $t$. We call $t$ the Schwinger parameter.
\end{lem}
\begin{proof}
  \begin{equation}
        \frac{1}{z} =\frac{\bar{z}}{z\bar{z}}\\
    =
\int_{0}^{+\infty}\bar{z}e^{-z\bar{z}t}dt\\
        =
        -\int_{0}^{+\infty}e^{-\frac{z\bar{z}}{t}}\bar{z}d\left(\frac{1}{t}\right)\\
        =
-\int_{0}^{+\infty}e^{-zy}dy.
\end{equation}
\end{proof}

We can use above lemma to rephrase the residue of a one-form $\alpha=\frac{g(z)}{z^{n}}dz$:
\begin{align*}
\oint_{z=0}\alpha 
&=
\oint_{z=0}\frac{g(z)}{z^{n}}dz \\
&=
\int_{(0,+\infty)^{n}}\int_{S^{1}}e^{-\left(\sum_{k=1}^{n}\frac{1}{t_{k}}\right)z\bar{z}}g(z)\prod_{k=1}^{n}dy_{k}dz,
\end{align*}
where $S^{1}$ is the set of unit length complex numbers and $y_{k}=\frac{\bar{z}}{t_{k}}$.

Let $d_{\mathbb{A}^{1}}$, $d_{t}$ be the de Rham differential on $\mathbb{A}^{1}$ and $(0,+\infty)^{n}$ respectively.
Then, the differential form appearing in the above formula is $(d_{\AA^{1}} + d_{t})$-closed:
$$
(d_{\mathbb{A}^{1}}+d_{t}) \left(e^{-\left(\sum_{k=1}^{n}\frac{1}{t_{k}}\right)z\bar{z}}g(z)\prod_{k=1}^{n}dy_{k}dz\right)=0.
$$
We will find an equivalent way to express the residue $\oint_{z=0} \alpha$ in terms of a homologous integration cycle.

Let 
$$S^{+}((0,+\infty)^n)=\{(t_{1},\cdots,t_{n})\in(0,+\infty)^{n}|\sum_{k=1}^{n}t_{k}^{2}=1\} \subset (0,+\infty)^{n},
$$ 
By Stokes' theorem, we obtain the following presentation for the residue  
\begin{align*}
\oint_{z=0}\alpha
&=\int_{(0,+\infty)^{n}}\int_{S^{1}}e^{-\left(\sum_{k=1}^{n}\frac{1}{t_{k}}\right)z\bar{z}}g(z)\prod_{k=1}^{n}dy_{k}dz\\
&=\int_{S^{+}((0,+\infty)^n)}\int_{\mathbb{A}^{1}}e^{-\left(\sum_{k=1}^{n}\frac{1}{t_{k}}\right)z\bar{z}}g(z)\prod_{k=1}^{n}dy_{k}dz.
\end{align*}
(In fact, the integration cycles $(0,\infty) \times S^1$ and $S^+((0,\infty)) \times \mathbb{A}^1$ are homologous.
This direct computation provides an explicit homotopy).
As a consequence, we have:
\begin{prop}
    Let $\alpha=\frac{g(z)}{z^{n}}dz$, we have
    $$
    \frac{1}{2\pi i}\int_{S^{+}((0,+\infty)^n)}\int_{\mathbb{A}^{1}}e^{-\left(\sum_{k=1}^{n}\frac{1}{t_{k}}\right)z\bar{z}}g(z)\prod_{k=1}^{n}dy_{k}dz=\frac{\partial_{z}^{n-1}g(0)}{(n-1)!}.
    $$
\end{prop}
\begin{proof}
    %Let's first compute
    %$$
    %\frac{1}{2\pi %i}\int_{\mathbb{A}^{1}}e^{-\left(\sum_{k=1}^{n}\f%rac{1}{t_{k}}\right)z\bar{z}}g(z)\prod_{k=1}^{n}%dy_{k}dz.
%    $$
Noticing the integral of Gaussian type, we can apply Wick's theorem to obtain
\begin{equation}
    \frac{1}{2\pi i}\int_{\mathbb{A}^{1}}e^{-\left(\sum_{k=1}^{n}\frac{1}{t_{k}}\right)z\bar{z}}g(z)\prod_{k=1}^{n}dy_{k}dz
   =(-1)^{n-1}\left.\frac{1}{\sum_{k=1}^{n}\frac{1}{t_{k}}}\iota_{\partial_{\bar{z}}}e^{\frac{1}{\sum_{k=1}^{n}\frac{1}{t_{k}}}\partial_{z}\partial_{\bar{z}}}\left(g(z)\prod_{k=1}^{n}dy_{k}\right)\right|_{z=0},
\end{equation}
where $\iota_{\partial_{\bar{z}}}$ denotes the operation of taking the interior product of the vector field $\partial_{\bar{z}}$ with a differential form.

Notice that as operators acting on differential forms, we have the following relation
\begin{align*}
    e^{\frac{1}{\sum_{k=1}^{n}\frac{1}{t_{k}}}\partial_{z}\partial_{\bar{z}}}\circ \left(dy_{i} \wedge - \right) \circ e^{\frac{-1}{\sum_{k=1}^{n}\frac{1}{t_{k}}}\partial_{z}\partial_{\bar{z}}}=dy_{i} \wedge +d(\frac{\frac{1}{t_{i}}}{\sum_{k=1}^{n}\frac{1}{t_{k}}})\partial_{z}.
\end{align*}
So, we can express the Gaussian integration as
\begin{align*}
    &\frac{1}{2\pi i}\int_{\mathbb{A}^{1}}e^{-\left(\sum_{k=1}^{n}\frac{1}{t_{k}}\right)z\bar{z}}g(z)\prod_{k=1}^{n}dy_{k}dz\\
    &=
    (-1)^{n-1}\sum_{k=1}^{n}\left((-1)^{k-1}\frac{\frac{1}{t_{k}}}{\sum_{k''=1}^{n}\frac{1}{t_{k''}}}\prod_{k'\neq k}d(\frac{\frac{1}{t_{k'}}}{\sum_{k''=1}^{n}\frac{1}{t_{k''}}})\right)\partial_{z}^{n-1}g(0).
\end{align*}

Therefore, 
\begin{align*}
     &\frac{1}{2\pi i}\int_{S^{+}((0,+\infty)^n)}\int_{\mathbb{A}^{1}}e^{-\left(\sum_{k=1}^{n}\frac{1}{t_{k}}\right)z\bar{z}}g(z)\prod_{k=1}^{n}dy_{k}dz\\
     &=
     (-1)^{n-1}\partial_{z}^{n-1}g(0)\int_{S^{+}((0,+\infty)^n)}\sum_{k=1}^{n}\left((-1)^{k-1}\frac{\frac{1}{t_{k}}}{\sum_{k''=1}^{n}\frac{1}{t_{k''}}}\prod_{k'\neq k}d(\frac{\frac{1}{t_{k'}}}{\sum_{k''=1}^{n}\frac{1}{t_{k''}}})\right).
\end{align*}

To compute the integral over $S^{+}((0,+\infty)^n)$, we apply localization to the following $\mathbb{R}^{+} = (0,+\infty)$-action on $(0,+\infty)^n$:
$$
\lambda\cdot (t_{1},\cdots,t_{n})=(\lambda t_{1},\cdots,\lambda t_{n}),\quad \lambda\in \mathbb{R}^{+}.
$$

We notice that
$$
\sum_{k=1}^{n}\left((-1)^{k-1}\frac{\frac{1}{t_{k}}}{\sum_{k''=1}^{n}\frac{1}{t_{k''}}}\prod_{k'\neq k}d(\frac{\frac{1}{t_{k'}}}{\sum_{k''=1}^{n}\frac{1}{t_{k''}}})\right)
$$
induces a well-defined differential form on $(0,+\infty)^{n}/\mathbb{R}^{+}$, so we have
\begin{align*}
&(-1)^{n-1}\int_{S^{+}((0,+\infty)^n)}\sum_{k=1}^{n}\left((-1)^{k-1}\frac{\frac{1}{t_{k}}}{\sum_{k''=1}^{n}\frac{1}{t_{k''}}}\prod_{k'\neq k}d(\frac{\frac{1}{t_{k'}}}{\sum_{k''=1}^{n}\frac{1}{t_{k''}}})\right)\\
&=
(-1)^{n-1}\int_{\{(t_{1},\cdots,t_{n})\in((0,+\infty)^n)|\sum_{k=1}^{n}\frac{1}{t_{k}}=1\}}\sum_{k=1}^{n}\left((-1)^{k-1}\frac{\frac{1}{t_{k}}}{\sum_{k''=1}^{n}\frac{1}{t_{k''}}}\prod_{k'\neq k}d(\frac{\frac{1}{t_{k'}}}{\sum_{k''=1}^{n}\frac{1}{t_{k''}}})\right)\\
&=
\int_{\{(u_{1},\cdots,u_{n})\in((0,+\infty)^n)|\sum_{k=1}^{n}u_{k}=1\}}\sum_{k=1}^{n}\left((-1)^{k-1}u_{k}\prod_{k'\neq k}du_{k'}\right)\\
&=\frac{1}{(n-1)!}.
\end{align*}

Thus, the conclusion follows.
\end{proof}

Although the expression
$$
\frac{1}{2\pi i}\int_{S^{+}((0,+\infty)^n)}\int_{\mathbb{A}^{1}}e^{-\left(\sum_{k=1}^{n}\frac{1}{t_{k}}\right)z\bar{z}}g(z)\prod_{k=1}^{n}dy_{k}dz
$$
is more complicate than the contour integral
$$
\frac{1}{2\pi i}\oint_{z=0}\frac{g(z)}{z^{n}}dz,
$$
we will show that it can be extended to higher dimensional affine spaces in next subsection.

\subsection{Higher residues.}

In this subsection, we generalize the concept of residues to the Jouanolou model of $(\omega_{\mathbb{A}^{d}}[d])^{\boxtimes \vec{I}}$, where $d\geq1$ is the dimension of the affine space, $\vec{I}=\{1<2<\cdots<n\}$ is an ordered set with $n$ elements.

Let $\mathcal{A}^{0,*}(\mathrm{Conf}_{\vec{I}}(\mathbb{A}^d))$ be the (algebraic) Dolbeault complex of the structure sheaf $\mathcal{O}(\mathrm{Conf}_{\vec{I}}(\mathbb{A}^{d}))$. Define the map
$$
i:\mathbf{J}_{\vec{I}}^{\mathbb{A}^{d}}\rightarrow\mathcal{A}^{0,*}(\mathrm{Conf}_{\vec{I}}(\mathbb{A}^d))
$$
$$
i(z_{i}^{s}) = z_{i}^{s},\quad i(x_{ij}^{s}) = \frac{\bar{z}_{i}^{s} - \bar{z}_{j}^{s}}{|z_{i} - z_{j}|^{2}},\quad i(\mathbf{d}x_{ij}^{s}) = \bar{\partial}\left(\frac{\bar{z}_{i}^{s} - \bar{z}_{j}^{s}}{|z_{i} - z_{j}|^{2}}\right).
$$
Here:
\begin{itemize}
    \item \(s \in \{1,2,\dots,d\}\) is the coordinate index.
    \item \(i, j \in \vec{I}\) are elements in the ordered set.
    \item \(|z_{i} - z_{j}|^{2} = \sum_{t=1}^{d}(z_{i}^{t} - z_{j}^{t})(\bar{z}_{i}^{t} - \bar{z}_{j}^{t})\) is the squared norm.
    \item \(\bar{\partial}\) is the Dolbeault differential.
\end{itemize}

It is easily checked that this map is well-defined and preserves the product of differential forms.
By construction, it is a cochain map.
Thus, $i$ is a morphism of commutative dg algebras.
(In fact, $i$ is injective, but this is not obvious and we will not make use of this fact).

We also point out that $i$ is a $\mathcal{D}_{(\mathbb{A}^{d})^{\vec{I}}}[(\mathbb{A}^{d})^{\vec{I}}]$-module morphism.
Similarly, we can define a natural dg $\mathcal{D}_{(\mathbb{A}^{d})^{\vec{I}}}[(\mathbb{A}^{d})^{\vec{I}}]$-module morphism from $\mathbf{J}_{\vec{I}}^{\mathbb{A}^{d}}((\omega_{\mathbb{A}^{d}}[d])^{\boxtimes\vec{I}})$ to $\mathcal{A}^{0,*}(\mathrm{Conf}_{\vec{I}}(\mathbb{A}^d),(\omega_{\mathbb{A}^{d}}[d])^{\boxtimes\vec{I}})$.

In the following discussion, we identify the elements of $\mathbf{J}_{\vec{I}}^{\mathbb{A}^{d}}((\omega_{\mathbb{A}^{d}}[d])^{\boxtimes\vec{I}})$ with the corresponding elements in $\mathcal{A}^{0,*}(\mathrm{Conf}_{\vec{I}}(\mathbb{A}^d),(\omega_{\mathbb{A}^{d}}[d])^{\boxtimes\vec{I}})$. 

In our previous discussion of residues when $d=1$, we represented the residue kernel $\frac{1}{z}$ as an integral over Schwinger parameters.
Now our goal is to represent $i(x_{ij}^{s})$ and $i(\mathbf{d}x_{ij}^{s})$ as integrals over Schwinger parameters:

\begin{lem}
    We have the following equalities:
    \begin{enumerate}
        \item 
        $$
        i(x_{ij}^{s})=-\int_{0}^{+\infty}e^{-z_{ij}\cdot y_{ij}}dy_{ij}^{s}
        $$
        \item 
        $$
        i(\mathbf{d}x_{ij}^{s})=-\int_{0}^{+\infty}e^{-z_{ij}\cdot y_{ij}}(z_{ij}\cdot dy_{ij})dy_{ij}^{s}
        $$
    \end{enumerate}
    Here:
        \begin{itemize}
            \item $y_{ij}^{s}=\frac{\bar{z}_{i}^{s}-\bar{z}_{j}^{s}}{t}$, where $t$ is called the Schwinger parameter.
            \item $z_{ij}^{s}=z_{i}^{s}-z_{j}^{s}$.
            \item 
            $z_{ij}\cdot y_{ij}=\sum_{s=1}^{d}z_{ij}^{s}y_{ij}^{s}$ is the dot product.
            \item $dy_{ij}^{s}$ is the differential form
            $$
            dy_{ij}^{s}=\frac{d\bar{z}_{i}^{s}-d\bar{z}_{j}^{s}}{t}-\frac{(\bar{z}_{i}^{s}-\bar{z}_{j}^{s})dt}{t^{2}}.
            $$
            \item the integrations are both with respect to the Schwinger parameter $t$.
        \end{itemize}
\end{lem}
\begin{proof}
    We note
    \begin{align*}
        i(x_{ij}^{s})
        &=
        \frac{\bar{z}_{i}^{s} - \bar{z}_{j}^{s}}{|z_{i} - z_{j}|^{2}}\\
        &=
        (\bar{z}_{i}^{s} - \bar{z}_{j}^{s})\int_{0}^{+\infty}e^{-|z_{i} - z_{j}|^{2}u}du\\
        &=
        -(\bar{z}_{i}^{s} - \bar{z}_{j}^{s})\int_{0}^{+\infty}e^{-\frac{|z_{i} - z_{j}|^{2}}{t}}d\left(\frac{1}{t}\right)\\
        &=
        -\int_{0}^{+\infty}e^{-z_{ij}\cdot y_{ij}}dy_{ij}^{s}.
    \end{align*}
    In the last line we have used the fact that only the $d t$ component contributes to the integral.

    By using dominated convergence theorem for derivatives, We can interchange the order of integration and differentiation when $z_{i}\neq z_{j}$. So
    \begin{align*}
        i(\mathbf{d}x_{ij}^{s})
        &=
        -\bar{\partial}\left(\int_{0}^{+\infty}e^{-z_{ij}\cdot y_{ij}}dy_{ij}^{s}\right)\\
        &=
        \int_{0}^{+\infty}\bar{\partial}\left(e^{-z_{ij}\cdot y_{ij}}dy_{ij}^{s}\right)\\
        &=
        \int_{0}^{+\infty}(\bar{\partial}+d_{t})\left(e^{-z_{ij}\cdot y_{ij}}dy_{ij}^{s}\right)\\
        &=
        -\int_{0}^{+\infty}e^{-z_{ij}\cdot y_{ij}}(z_{ij}\cdot dy_{ij})dy_{ij}^{s},
    \end{align*}
    where $d_t$ is the de Rham differential over the Schwinger parameter space.
    Again, we emphasize that only the term proportional to $dt$ will contribute to the integral.
\end{proof}

The following proposition is useful and is a direct computation.

\begin{prop}\label{Lie and interior product}
    For $i,j \in \vec{I}$ define the vector field
    $$ V_{ij}=\sum_{s=1}^{d}(\bar{z}_i^s\partial_{\bar{z}_{i}^{s}}+\bar{z}_j^s\partial_{\bar{z}_{j}^{s}})+t\partial_{t}.
    $$
    If $L_{\vec{V_{ij}}}$ and $\iota_{\vec{V_{ij}}}$ denote the Lie derivative and interior product operator, respectively, then we have the following identities:
$$
\left\{
\begin{array}{cc}
     L_{\vec{V}_{ij}}e^{-z_{ij}\cdot y_{ij}}dy_{ij}^s & =0,  \\
      L_{\vec{V}_{ij}}e^{-z_{ij}\cdot y_{ij}}(z_{ij}\cdot y_{ij})dy_{ij}^s & =0,\\
      \iota_{\vec{V}_{ij}}e^{-z_{ij}\cdot y_{ij}}dy_{ij}^s & =0,\\
      \iota_{\vec{V}_{ij}}e^{-z_{ij}\cdot y_{ij}}(z_{ij}\cdot y_{ij})dy_{ij}^s & =0.
\end{array}
\right.
$$
\end{prop}
%\begin{proof}
%    These identities follow from direct computations.
%\end{proof}

%With these prelimilinary computations in order, we can now explain our strategy for developing residue theory based on Jouanolou’s model of higher-dimensional affine spaces.
%Let
%$$
%\alpha=p(x_{ij}^{s},\mathbf{d}x_{ij}^{s})\otimes \beta\in \mathbf{J}_{\vec{I}}^{\mathbb{A}^{d}}\otimes_{\mathcal{O}((\mathbb{A}^{d})^{\vec{I}})}(\omega_{\mathbb{A}^{d}}[d])^{\boxtimes\vec{I}}[(\mathbb{A}^{d})^{\vec{I}}]\cong\mathbf{J}_{\vec{I}}^{\mathbb{A}^{d}}((\omega_{\mathbb{A}^{d}}[d])^{\boxtimes\vec{I}}),
%$$
%where $p(x_{ij}^{s},\mathbf{d}x_{ij}^{s})$ is a polynomial with variables $\{x_{ij}^{s},\mathbf{d}x_{ij}^{s}\}_{i<j\in\vec{I},1\leq s\leq d}$.
%Recall that $\mathbf{d} x_{ij}^{s}$ are odd variables with respect to the Koszul rule of sign, so they anti-commute with one another.
%We take $p$ to be homogenous of total degree $l$.
%We want to define the residue by the following formula:
%\begin{align*}
%    &\oint_{z_1,\dots,z_{n-1}=z_n}p(x_{ij}^{s},\mathbf{d}x_{ij}^{s})\otimes \beta\\
%    &=\pm\int_{S^{+}((0,+\infty)^l)}\int_{(\mathbb{A}^{d})^{\vec{I}-\{n\}}}p(-e^{-z_{ij}\cdot y_{ij}}dy_{ij}^{s},-e^{-z_{ij}\cdot y_{ij}}(z_{ij}\cdot dy_{ij})dy_{ij}^{s})\otimes \beta,
%\end{align*}
%where 
%$$
%S^{+}((0,+\infty)^l)=\{(t_1,\dots,t_{l})\in(0,+\infty)^l|\sum_{i=1}^{l}t_{i}^{2}=1\}.
%$$
%
%In this subsection, we will prove the following facts (some in their stronger forms):
%\begin{enumerate}
%    \item When $t_{l'}\in(0,+\infty)$ for $l'\in\{1,2,\dots,l\}$,
%    $$
%    p(-e^{-z_{ij}\cdot y_{ij}}dy_{ij}^{s},-e^{-z_{ij}\cdot y_{ij}}(z_{ij}\cdot dy_{ij})dy_{ij}^{s})\otimes \beta
%    $$
%    is integrable over $(\mathbb{A}^{d})^{\vec{I}-\{n\}}$.
%    \item 
%    $$\int_{(\mathbb{A}^{d})^{\vec{I}-\{n\}}}p(-e^{-z_{ij}\cdot y_{ij}}dy_{ij}^{s},-e^{-z_{ij}\cdot y_{ij}}(z_{ij}\cdot dy_{ij})dy_{ij}^{s})\otimes \beta
%    $$
%    is integrable over $S^{+}((0,+\infty)^l)$. So $\oint_{z_1,\dots,z_{n-1}=z_n}p(x_{ij}^{s},\mathbf{d}x_{ij}^{s})\otimes \beta$ is well-defined as an integral.
%    \item If 
%    $$
%    \alpha=p(x_{ij}^{s},\mathbf{d}x_{ij}^{s})\otimes \beta=p'(x_{ij}^{s},\mathbf{d}x_{ij}^{s})\otimes \beta'\in \mathbf{J}_{\vec{I}}^{\mathbb{A}^{d}}((\omega_{\mathbb{A}^{d}}[d])^{\boxtimes\vec{I}}),
%    $$
%    we have
%    $$
%    \oint_{z_1,\dots,z_{n-1}=z_n}p(x_{ij}^{s},\mathbf{d}x_{ij}^{s})\otimes \beta=\oint_{z_1,\dots,z_{n-1}=z_n}p'(x_{ij}^{s},\mathbf{d}x_{ij}^{s})\otimes \beta'.
%    $$
%    So the residues are well-defined on $\mathbf{J}_{\vec{I}}^{\mathbb{A}^{d}}((\omega_{\mathbb{A}^{d}}[d])^{\boxtimes\vec{I}})$.
%    \item Let $s\in\{1,\dots,d\}$, $i\in\{1<2<\cdots<n\}$,
%    $$
%    \oint_{z_1,\dots,z_{n-1}=z_n}\alpha\cdot \partial_{z_{i}^{s}}=
%    \begin{cases}
%        0, &\text{when  }i\neq n,\\
%    \left(\oint_{z_1,\dots,z_{n-1}=z_n}\alpha\right)\cdot \partial_{z_{i}^{s}}, &\text{when  }i=n.
%    \end{cases}
%    $$
%    This result generalizes the corresponding result of ordinary residues.
%\end{enumerate}

We introduce a graphical description for elements in $\mathbf{J}_{\vec{I}}^{\mathbb{A}^{d}}((\omega_{\mathbb{A}^{d}}[d])^{\boxtimes\vec{I}})$.
\begin{defn}
  A \textit{directed graph} $\vec{\Gamma}$ consists of the following data:
    \begin{enumerate}
        \item An ordered set $\vec{\Gamma}_{0}$ of vertices. We use $|\Gamma_{0}|$ to denote the number of vertices.
        \item An ordered set $\vec{\Gamma}_{0}$ of directed edges. We use $|\Gamma_{1}|$ to denote the number of directed edges.
        \item Two maps
        $$
        t,h\colon \Gamma_{1}\rightarrow\Gamma_{0},
        $$
        which are assignments of tail and head to each directed edge. We require that 
        $$
        t(e)\neq h(e)
        $$
        for any $e\in\vec{\Gamma}_{1}$, i.e., the graph $\vec{\Gamma}$ has no self-loops.
    \end{enumerate}
    A \textit{decoration} on a directed graph $\vec{\Gamma}$ is the choice of a special edge $e_{l}\in \vec{\Gamma}_{1}$ and a map 
    $$
    m\colon \Gamma_{1}\rightarrow\{1,2,\dots,d\}.
    $$
\end{defn}

We will use $(\vec{\Gamma},m,l)$ to denote a decorated directed graph. 
We refer to a decorated directed graph as a \textit{Feynman graph} in what follows.

For a directed graph $\vec{\Gamma}$, we use the following notations to describe $\vec{\Gamma}_{0}$ and $\vec{\Gamma}_{1}$:
$$
\begin{cases}
    \vec{\Gamma}_{0}=\{1<\cdots<n=|\Gamma_{0}|\},\\
    \vec{\Gamma}_{1}=\{e_{1}<\cdots<e_{|\Gamma_{1}|}\}.
\end{cases}
$$
\begin{defn}
    Let $(\vec{\Gamma},m,l)$ be a Feynman graph. 
    Define
    $$
    p_{(\vec{\Gamma},m,l)}(x_{e}^{m(e)},\mathbf{d}x_{e}^{m(e)})=\prod_{e<e_{l}\in\vec{\Gamma}_{1}}x_{e}^{m(e)}\prod_{e_l\leq e'\in\vec{\Gamma}_{1}}\mathbf{d}x_{e'}^{m(e')}\in k[x_{ij}^{s},\mathbf{d}x_{ij}^{s}]_{i<j\in\vec{\Gamma}_{0}}^{1\leq s\leq d},
    $$
    where the order of the factors is determined by the order of $\vec{\Gamma}_{1}$. We have used the convention
    $$
    x_{e}^{m(e)}=x_{t(e)h(e)}^{m(e)}=-x_{h(e)t(e)}^{m(e)}.
    $$
    We call $p_{(\vec{\Gamma},m,l)}$ the \textit{monomial corresponding to} $(\vec{\Gamma},m,l)$.
\end{defn}

\begin{rem}
    $\{p_{(\vec{\Gamma},m,l)}\}$ spans $k[x_{ij}^{s},\mathbf{d}x_{ij}^{s}]_{i<j\in\vec{\Gamma}_{0}}^{1\leq s\leq d}$, so we can use Feynman graphs to represent elements in $k[x_{ij}^{s},\mathbf{d}x_{ij}^{s}]_{i<j\in\vec{\Gamma}_{0}}^{1\leq s\leq d}$.
\end{rem}

We now introduce some concepts from graph theory.
See appendix \ref{graph theory} for more details and explanation.

\begin{defn}
    Let $\vec{\Gamma}$ be a directed graph. The incidence matrix $\rho=(\rho_{ei})_{e\in\vec{\Gamma}_{1},i\in\vec{\Gamma}_{0}}$ is given by
    $$
    \rho_{ei}=
    \begin{cases}
        1, &\text{if }t(e)=i,\\
        -1, &\text{if }s(e)=i,\\
        0, &\text{otherwise.}
    \end{cases}
    $$
\end{defn}
\begin{defn}
    Let $\vec{\Gamma}$ be a directed graph. For any $e\in\vec{\Gamma}_{1}$, $t_{e}\in(0,+\infty)$. The weighted Laplacian matrix is given by
    $$
    M_{\vec{\Gamma}}(t)_{ij}=
    \sum_{e\in\vec{\Gamma}_{1}}\rho_{ei}\frac{1}{t_{e}}\rho_{ej}.
    $$
\end{defn}
\begin{prop}\label{Minverse}
    Let $\vec{\Gamma}$ be a directed graph. For any $e\in\vec{\Gamma}_{1}$, $t_e\in(0,+\infty)$. If $\vec{\Gamma}$ is a connected graph, then the matrix $M_{\vec{\Gamma}}(t)=(M_{\vec{\Gamma}}(t)_{ij})_{i,j\in\vec{\Gamma}_{0}-\{n\}}$ is invertible. We use $M^{-1}_{\vec{\Gamma}}(t)=(M^{-1}_{\vec{\Gamma}}(t)_{ij})_{i,j\in\vec{\Gamma}_{0}-\{n\}}$ to denote the inverse matrix.
\end{prop}
\begin{proof}
    See corollary \ref{det of laplacian}.
\end{proof}
\begin{defn}
    Let $\vec{\Gamma}$ be a connected directed graph. For any $e\in\vec{\Gamma}_{1}$, $t_e\in(0,+\infty)$. The graphic Green's function $d^{-1}_{\vec{\Gamma}}(t)=(d^{-1}_{\vec{\Gamma}}(t)_{ei})_{e\in\vec{\Gamma}_{1},i\in\vec{\Gamma}_{0}-\{n\}}$ is given by
    $$
    d^{-1}_{\vec{\Gamma}}(t)_{ei}=\sum_{j=1}^{n-1}\frac{1}{t_{e}}\rho_{ej}M^{-1}_{\vec{\Gamma}}(t)_{ji}.
    $$
\end{defn}

%We collect some facts in graph theory in appendix \ref{graph theory}.

We define the class of integrals which interested us.

\begin{defn}
    Let $(\vec{\Gamma},m,l)$ be a Feynman graph, such that $\vec{\Gamma}_{0}=\vec{I}$. Given 
    $$
    \beta\in (\omega_{\mathbb{A}^{d}}[d])^{\boxtimes\vec{I}}[(\mathbb{A}^{d})^{\vec{I}}]
    $$
    and
    $$
    e^{\sum_{i=1}^{n}z_{i}\cdot\lambda_{i}}=e^{\sum_{s=1}^{d}\sum_{i=1}^{n}z^{s}_{i}\lambda^{s}_{i}}\in C^{\infty}((\mathbb{A}^{d})^{\vec{I}}\times (\mathbb{A}^{d})^{\vec{I}}),
    $$
    the \textit{Feynman graph integrand} is defined by 
    \begin{align*}
        &W((\vec{\Gamma},m,l),\beta e^{\sum_{i=1}^{n}z_{i}\cdot\lambda_{i}})\\
        &=
        \int_{(\mathbb{A}^{d})^{\vec{I}-\{n\}}}p_{(\vec{\Gamma},m,l)}(-e^{-z_{e}\cdot y_{e}}dy_{e}^{m(e)},-e^{-z_{e}\cdot y_{e}}(z_{e}\cdot dy_{e})dy_{e}^{m(e)})e^{\sum_{i=1}^{n}z_{i}\cdot\lambda_{i}}\otimes \beta,
    \end{align*}
    where $z_{e}=z_{t(e)}-z_{h(e)}$, $y_{e}=\frac{\bar{z}_{t(e)}-\bar{z}_{h(e)}}{t_{e}}$.
\end{defn}

We introduce the following coordinates on $(\mathbb{A}^{d})^{\vec{I}}$:
$$
\begin{cases}
    z_{i}=\tilde{z}_{i}+\tilde{z}_{n}, &\text{if }i\neq n,\\
    z_{n}=\tilde{z}_{n}.
\end{cases}
$$

Now, we can get the following result:
\begin{prop}\label{explicit formula}
    When $t_{e}\in(0,+\infty)$ for $e\in\vec{\Gamma}_{1}$,
    $$
    W((\vec{\Gamma},m,l),\beta e^{\sum_{i=1}^{n}z_{i}\cdot\lambda_{i}})
    $$
    is a convergent integral. Moreover,
    \begin{enumerate}
        \item If $\vec{\Gamma}$ is disconnected,
        $$
        W((\vec{\Gamma},m,l),\beta e^{\sum_{i=1}^{n}z_{i}\cdot\lambda_{i}})=0.
        $$
        \item If $\vec{\Gamma}$ is connected, we have the following explicit formula:
        \begin{align*}
            &W((\vec{\Gamma},m,l),\beta e^{\sum_{i=1}^{n}z_{i}\cdot\lambda_{i}})\\
            &=(-2\pi i)^{d(n-1)}\left(
        \iota_{\prod_{i=1}^{n-1}(d^{d}\tilde{z}_{i}d^{d}\bar{\tilde{z}}_{i})}\circ p_{(\vec{\Gamma},m,l)}(-d\hat{y}_{e}^{m(e)},-(\sum_{s=1}^{d}(\partial_{\lambda_{e}^{s}}+\tilde{z}_{e}^{s})\circ d\hat{y}_{e}^{s})\circ d\hat{y}_{e}^{m(e)})
        \right.\\
        &\left.\left.(1\otimes \beta)\right)\right|_{\tilde{z}_{i}=0,1\leq i\leq n-1} e^{\tilde{z}_{n}\cdot\sum_{1}^{n}\lambda_{i}}.
        \end{align*}
    \end{enumerate}
  \end{prop}

  The notations appearing in the above proposition are explained as follows:
    \begin{itemize}
        \item $\iota_{\prod_{i=1}^{n-1}(d^{d}\tilde{z}_{i}d^{d}\bar{\tilde{z}}_{i})}$: if $
        \gamma\in \mathcal{A}^{d(n-1),d(n-1)}((\mathbb{A}^d)^{\vec{I}-\{n\}})
        $ is a top form, there exists a unique
        $
        \gamma'\in C^{\infty}((\mathbb{A}^d)^{\vec{I}-\{n\}}),
        $
        such that
        $
        \gamma=\gamma'\prod_{i=1}^{n-1}(d^{d}\tilde{z}_{i}d^{d}\bar{\tilde{z}}_{i}).
        $
        We define 
        $$
        \iota_{\prod_{i=1}^{n-1}(d^{d}\tilde{z}_{i}d^{d}\bar{\tilde{z}}_{i})}\gamma=\gamma'.
        $$
        If $\gamma$ is not a top form, we define
        $$
        \iota_{\prod_{i=1}^{n-1}(d^{d}\tilde{z}_{i}d^{d}\bar{\tilde{z}}_{i})}\gamma=0.
        $$
        \item $d\hat{y}_{e}^{s}$: it is an operator defined by
        $$
        d\hat{y}_{e}^{s}=\sum_{j=1}^{n-1}\left(d^{-1}_{\vec{\Gamma}}(t)_{ej}d\bar{\tilde{z}}_{j}^{s}+d(d^{-1}_{\vec{\Gamma}}(t)_{ej})(\partial_{\tilde{z}_{j}^{s}}+\lambda_{j}^{s})\right),
        $$
        where $e\in \vec{\Gamma}_{1}$, $1\leq s\leq d$.
    \end{itemize}

    \begin{proof}[Proof of the proposition]
    To simplify notations, we define
    $$
    \tilde{W}=p_{(\vec{\Gamma},m,l)}(-e^{-z_{e}\cdot y_{e}}dy_{e}^{m(e)},-e^{-z_{e}\cdot y_{e}}(z_{e}\cdot dy_{e})dy_{e}^{m(e)})e^{\sum_{i=1}^{n}z_{i}\cdot\lambda_{i}}\otimes \beta.
    $$

    We first prove (1). If $\vec{\Gamma}$ is disconnected, let $\vec{\Gamma}'$ be a connected component of $\vec{\Gamma}$, such that $n\notin \vec{\Gamma}'$. Let $\vec{V}=\sum_{i\in\vec{\Gamma}'_{0}}\partial_{\bar{z}_{i}}$, $\iota_{\vec{V}}$ is the interior product operator of $\vec{V}$. We can check that
    $$
    \iota_{\vec{V}}(\tilde{W})=0,
    $$
    so $\tilde{W}$ is not a top form on $(\mathbb{A}^d)^{\vec{I}-\{n\}}$. Hence $$
    W((\vec{\Gamma},m,l),\beta e^{\sum_{i=1}^{n}z_{i}\cdot\lambda_{i}})=\int_{(\mathbb{A}^{d})^{\vec{I}-\{n\}}}\tilde{W}=0.
    $$

    Now, let's prove (2).
    We begin by using Wick's theorem to explicitly evaluate the Gaussian integrals.
    If $\vec{\Gamma}$ is connected, by using coordinates $\{\tilde{z}_{i}\}$, we have
    \begin{align*}
        \tilde{W}&=e^{-\sum_{i=1}^{n-1}\sum_{j=1}^{n-1}\tilde{z}_{i}\cdot M_{\vec{\Gamma}}(t)_{ij}\bar{\tilde{z}}_{j}}p_{(\vec{\Gamma},m,l)}(-dy_{e}^{m(e)},-(\tilde{z}_{e}\cdot dy_{e})dy_{e}^{m(e)})e^{\sum_{i=1}^{n-1}\tilde{z}_{i}\cdot\lambda_{i}+\tilde{z}_{n}\cdot\sum_{1}^{n}\lambda_{i}}\otimes \beta.
    \end{align*}
    Since $\vec{\Gamma}$ is connected, by Proposition \ref{Minverse}, $M_{\vec{\Gamma}}(t)$ is invertible. We notice that the integral over $(\mathbb{A}^d)^{\vec{I}-\{n\}}$ is a Gaussian type integral, so
    \begin{align*}
        &W((\vec{\Gamma},m,l),\beta e^{\sum_{i=1}^{n}z_{i}\cdot\lambda_{i}})\\
        &=\int_{(\mathbb{A}^d)^{\vec{I}-\{n\}}}e^{-\sum_{i=1}^{n-1}\sum_{j=1}^{n-1}\tilde{z}_{i}\cdot M_{\vec{\Gamma}}(t)_{ij}\bar{\tilde{z}}_{j}}\prod_{i=1}^{n-1}(d^{d}\tilde{z}_{i}d^{d}\bar{\tilde{z}}_{i})\wedge \iota_{\prod_{i=1}^{n-1}(d^{d}\tilde{z}_{i}d^{d}\bar{\tilde{z}}_{i})}\\
        &\left(p_{(\vec{\Gamma},m,l)}(-dy_{e}^{m(e)},-(\tilde{z}_{e}\cdot dy_{e})dy_{e}^{m(e)})e^{\sum_{i=1}^{n-1}\tilde{z}_{i}\cdot\lambda_{i}+\tilde{z}_{n}\cdot\sum_{1}^{n}\lambda_{i}}\otimes \beta\right)\\
        &=(-2\pi i)^{d(n-1)}\frac{1}{(detM_{\vec{\Gamma}(t)})^{d}}\left(
        \iota_{\prod_{i=1}^{n-1}(d^{d}\tilde{z}_{i}d^{d}\bar{\tilde{z}}_{i})}
        \circ
        e^{\sum_{s=1}^{d}\sum_{i=1}^{n-1}\sum_{j=1}^{n-1} M^{-1}_{\vec{\Gamma}}(t)_{ij}\partial_{\tilde{z}_{i}^{s}}\partial_{\bar{\tilde{z}}_{j}^{s}}}
        \right.\\
        &\left.\left.p_{(\vec{\Gamma},m,l)}(-dy_{e}^{m(e)},-(\tilde{z}_{e}\cdot dy_{e})dy_{e}^{m(e)})e^{\sum_{i=1}^{n-1}\tilde{z}_{i}\cdot\lambda_{i}+\tilde{z}_{n}\cdot\sum_{1}^{n}\lambda_{i}}\otimes \beta\right)\right|_{\tilde{z}_{i}=0,1\leq i\leq n-1}.
    \end{align*}

    Introduce the new basis for the space of one-forms:
    $$
    \begin{cases}
        d\tilde{z}_{i}=d\tilde{z}_{i}&\text{for }1\leq i\leq n,\\
        d\bar{\tilde{z}}_{i}=\sum_{j=1}^{n-1}d\left(
        M^{-1}_{\vec{\Gamma}(t)_{ij}}\bar{z'}_{j}
        \right)=\sum_{j=1}^{n-1}\left(
        d(M^{-1}_{\vec{\Gamma}(t)_{ij}})\bar{z'}_{j}+M^{-1}_{\vec{\Gamma}(t)_{ij}}d\bar{z'}_{j}
        \right) &\text{for }1\leq i\leq n-1,\\
        d\bar{\tilde{z}}_{n}=d\bar{\tilde{z}}_{n}\\
        dt_{e}=dt_{e}&\text{for }e\in\vec{\Gamma}_{1},
    \end{cases}
    $$
    where $\bar{z'}_{i}$ is a smooth function defined by
    $$
    \bar{z'}_{i}=\sum_{j=1}^{n-1}M_{\vec{\Gamma}(t)_{ij}}\bar{z}_{j}.
    $$
    The dual basis for vectors transforms as follows:
    $$
    \begin{cases}
        \partial_{\tilde{z}_{i}}\rightarrow \partial_{\tilde{z}_{i}}&\text{for }1\leq i\leq n,\\
        \partial_{\bar{\tilde{z}}_{i}}\rightarrow \sum_{j=1}^{n-1}
        M_{\vec{\Gamma}(t)_{ij}}\partial_{\bar{z'}_{j}}
         &\text{for }1\leq i\leq n-1,\\
        \partial_{\bar{\tilde{z}}_{n}}\rightarrow \partial_{\bar{\tilde{z}}_{n}}\\
        \partial_{t_{e}}=\partial_{t_{e}}-\sum_{i,j,k\in \vec{I}-\{n\}}M_{\vec{\Gamma}(t)_{ik}}\partial_{t_{e}}(M^{-1}_{\vec{\Gamma}(t)_{kj}})\bar{z'}_{j}\partial_{\bar{z'}_{i}}&\text{for }e\in\vec{\Gamma}_{1}.
        
    \end{cases}
    $$

    Notice that 
    $$
    \iota_{\prod_{i=1}^{n-1}(d^{d}\tilde{z}_{i}d^{d}\bar{\tilde{z}}_{i})}=(\det M_{\vec{\Gamma}(t)})^{d}\iota_{\prod_{i=1}^{n-1}(d^{d}\tilde{z}_{i}d^{d}\bar{z'}_{i})},
    $$
    so the factors involving determinants cancel and the Feynman graph integrand becomes
    \begin{align*}
        &W((\vec{\Gamma},m,l),\beta e^{\sum_{i=1}^{n}z_{i}\cdot\lambda_{i}})\\
        &=(-2\pi i)^{d(n-1)}\left(
        \iota_{\prod_{i=1}^{n-1}(d^{d}\tilde{z}_{i}d^{d}\bar{z'}_{i})}
        \circ
        e^{\sum_{s=1}^{d}\sum_{i=1}^{n-1}\partial_{\tilde{z}_{i}^{s}}\partial_{\bar{z'}_{i}^{s}}}
        \right.\\
        &\left.\left.p_{(\vec{\Gamma},m,l)}(-dy_{e}^{m(e)},-(\tilde{z}_{e}\cdot dy_{e})dy_{e}^{m(e)})e^{\sum_{i=1}^{n-1}\tilde{z}_{i}\cdot\lambda_{i}+\tilde{z}_{n}\cdot\sum_{1}^{n}\lambda_{i}}\otimes \beta\right)\right|_{\tilde{z}_{i}=0,1\leq i\leq n-1}.
    \end{align*}

    Now, we have obtained a better formula for the Feynman graph integral. However, the appearance of exponential
    functions $e^{\sum_{s=1}^{d}\sum_{i=1}^{n-1}\partial_{\tilde{z}_{i}^{s}}\partial_{\bar{z'}_{i}^{s}}}$ and
    $e^{\sum_{i=1}^{n-1}\tilde{z}_{i}\cdot\lambda_{i}+\tilde{z}_{n}\cdot\sum_{1}^{n}\lambda_{i}}$ make this formula
    unsuitable for an algebraic setting and we must further simplify. 
    We notice that
    $$
    \tilde{z}^{s}_{e}e^{\sum_{i=1}^{n-1}\tilde{z}_{i}\cdot\lambda_{i}+\tilde{z}_{n}\cdot\sum_{1}^{n}\lambda_{i}}=\partial_{\lambda_{e}^{s}}e^{\sum_{i=1}^{n-1}\tilde{z}_{i}\cdot\lambda_{i}+\tilde{z}_{n}\cdot\sum_{1}^{n}\lambda_{i}}
    $$
    Using this, we obtain  
    \begin{align*}
        &p_{(\vec{\Gamma},m,l)}(-dy_{e}^{m(e)},-(\tilde{z}_{e}\cdot dy_{e})dy_{e}^{m(e)})e^{\sum_{i=1}^{n-1}\tilde{z}_{i}\cdot\lambda_{i}+\tilde{z}_{n}\cdot\sum_{1}^{n}\lambda_{i}}\\
        &=
        p_{(\vec{\Gamma},m,l)}(-dy_{e}^{m(e)},-(\sum_{s=1}^{d}\partial_{\lambda_{e}^{s}}\circ dy_{e}^{s})\circ dy_{e}^{m(e)})e^{\sum_{i=1}^{n-1}\tilde{z}_{i}\cdot\lambda_{i}+\tilde{z}_{n}\cdot\sum_{1}^{n}\lambda_{i}}.
    \end{align*}
    We also notice that
    \begin{align*}
        e^{\sum_{s=1}^{d}\sum_{i=1}^{n-1}\partial_{\tilde{z}_{i}^{s}}\partial_{\bar{z'}_{i}^{s}}}\circ dy_{e}^{s}=d\tilde{y}_{e}^{s}\circ
        e^{\sum_{s=1}^{d}\sum_{i=1}^{n-1}\partial_{\tilde{z}_{i}^{s}}\partial_{\bar{z'}_{i}^{s}}},
    \end{align*}
    where 
    $$
    d\tilde{y}_{e}^{s}=\sum_{j=1}^{n-1}\left(
        d\left(d^{-1}_{\vec{\Gamma}}(t)_{ej}\bar{z'}_{j}^{s}\right)+d\left(d^{-1}_{\vec{\Gamma}}(t)_{ej}\right)\partial_{\tilde{z}_{j}^{s}}
        \right)
    $$
    is an operator. So we have
    \begin{align*}
        &e^{\sum_{s=1}^{d}\sum_{i=1}^{n-1}\partial_{\tilde{z}_{i}^{s}}\partial_{\bar{z'}_{i}^{s}}}\circ
        p_{(\vec{\Gamma},m,l)}(-dy_{e}^{m(e)},-(\sum_{s=1}^{d}\partial_{\lambda_{e}^{s}}\circ dy_{e}^{s})\circ dy_{e}^{m(e)})\\
        &=
        p_{(\vec{\Gamma},m,l)}(-d\tilde{y}_{e}^{m(e)},-(\sum_{s=1}^{d}\partial_{\lambda_{e}^{s}}\circ d\tilde{y}_{e}^{s})\circ d\tilde{y}_{e}^{m(e)})\circ 
        e^{\sum_{s=1}^{d}\sum_{i=1}^{n-1}\partial_{\tilde{z}_{i}^{s}}\partial_{\bar{z'}_{i}^{s}}}.
    \end{align*}
    The Feynman graph integrand becomes
    \begin{align*}
        &W((\vec{\Gamma},m,l),\beta e^{\sum_{i=1}^{n}z_{i}\cdot\lambda_{i}})\\
        &=(-2\pi i)^{d(n-1)}\left(
        \iota_{\prod_{i=1}^{n-1}(d^{d}\tilde{z}_{i}d^{d}\bar{z'}_{i})}\circ p_{(\vec{\Gamma},m,l)}(-d\tilde{y}_{e}^{m(e)},-(\sum_{s=1}^{d}\partial_{\lambda_{e}^{s}}\circ d\tilde{y}_{e}^{s})\circ d\tilde{y}_{e}^{m(e)})
        \right.\\
        &\left.\left.\circ e^{\sum_{i=1}^{n-1}\tilde{z}_{i}\cdot\lambda_{i}+\tilde{z}_{n}\cdot\sum_{1}^{n}\lambda_{i}}(1\otimes \beta)\right)\right|_{\tilde{z}_{i}=0,1\leq i\leq n-1}.
    \end{align*}
    We further notice that
    $$
    d\tilde{y}_{e}^{s}\circ e^{\sum_{i=1}^{n-1}\tilde{z}_{i}\cdot\lambda_{i}+\tilde{z}_{n}\cdot\sum_{1}^{n}\lambda_{i}}=e^{\sum_{i=1}^{n-1}\tilde{z}_{i}\cdot\lambda_{i}+\tilde{z}_{n}\cdot\sum_{1}^{n}\lambda_{i}}\circ d\tilde{\tilde{y}}_{e}^{s},
    $$
    where 
    $$
    d\tilde{\tilde{y}}_{e}^{s}=d\tilde{y}_{e}^{s}+\sum_{j=1}^{n-1}d\left(d^{-1}_{\vec{\Gamma}}(t)_{ej}\right)\lambda_{j}^{s}.
    $$
    Then we have
    \begin{align*}
        &p_{(\vec{\Gamma},m,l)}(-d\tilde{y}_{e}^{m(e)},-(\sum_{s=1}^{d}\partial_{\lambda_{e}^{s}}\circ d\tilde{y}_{e}^{s})\circ d\tilde{y}_{e}^{m(e)})\circ e^{\sum_{i=1}^{n-1}\tilde{z}_{i}\cdot\lambda_{i}+\tilde{z}_{n}\cdot\sum_{1}^{n}\lambda_{i}}\\
        &=e^{\sum_{i=1}^{n-1}\tilde{z}_{i}\cdot\lambda_{i}+\tilde{z}_{n}\cdot\sum_{1}^{n}\lambda_{i}}\circ p_{(\vec{\Gamma},m,l)}(-d\tilde{\tilde{y}}_{e}^{m(e)},-(\sum_{s=1}^{d}(\partial_{\lambda_{e}^{s}}+\tilde{z}_{e}^{s})\circ d\tilde{\tilde{y}}_{e}^{s})\circ d\tilde{\tilde{y}}_{e}^{m(e)}).
    \end{align*}

    Finally, we get
    \begin{align*}
        &W((\vec{\Gamma},m,l),\beta e^{\sum_{i=1}^{n}z_{i}\cdot\lambda_{i}})\\
        &=(-2\pi i)^{d(n-1)}\left(
        \iota_{\prod_{i=1}^{n-1}(d^{d}\tilde{z}_{i}d^{d}\bar{z'}_{i})}\circ p_{(\vec{\Gamma},m,l)}(-d\tilde{\tilde{y}}_{e}^{m(e)},-(\sum_{s=1}^{d}(\partial_{\lambda_{e}^{s}}+\tilde{z}_{e}^{s})\circ d\tilde{\tilde{y}}_{e}^{s})\circ d\tilde{\tilde{y}}_{e}^{m(e)})
        \right.\\
        &\left.\left.(1\otimes \beta)\right)\right|_{\tilde{z}_{i}=0,1\leq i\leq n-1} e^{\tilde{z}_{n}\cdot\sum_{1}^{n}\lambda_{i}}\\
        &=(-2\pi i)^{d(n-1)}\left(
        \iota_{\prod_{i=1}^{n-1}(d^{d}\tilde{z}_{i}d^{d}\bar{\tilde{z}}_{i})}\circ p_{(\vec{\Gamma},m,l)}(-d\hat{y}_{e}^{m(e)},-(\sum_{s=1}^{d}(\partial_{\lambda_{e}^{s}}+\tilde{z}_{e}^{s})\circ d\hat{y}_{e}^{s})\circ d\hat{y}_{e}^{m(e)})
        \right.\\
        &\left.\left.(1\otimes \beta)\right)\right|_{\tilde{z}_{i}=0,1\leq i\leq n-1} e^{\tilde{z}_{n}\cdot\sum_{1}^{n}\lambda_{i}}.\\
    \end{align*}
\end{proof}

Now, we want to prove the Feynman graph integrand is integrable over 
$$
S^{+}((0,+\infty)^{\vec{\Gamma}_{1}})=\{(t_{e_{1}},\dots,t_{e_{|\Gamma_{1}|}})\in(0,+\infty)^{\vec{\Gamma}_{1}}|\sum_{i=1}^{|\Gamma_{1}|}t_{i}^{2}=1\}.
$$
We use the compactification technique of Schwinger spaces as developed in \cite{wang2024feynman}.
For more details see appendix \ref{Schwinger spaces}.

    Let $\vec{\Gamma}$ be a directed graph, $(0,+\infty)^{\vec{\Gamma}_{1}}$ is called the Schwinger space of
    $\vec{\Gamma}$. The orientation on Schwinger space is determined by
    $$
    \int_{(0,L)^{\vec{\Gamma}_{1}}}\prod_{e\in\vec{\Gamma}_{1}}dt_{e}=L^{|\Gamma_{1}|},
    $$
    where $L>0$.

There is a natural partial compactification of $(0,+\infty)^{\vec{\Gamma}_{1}}$, which is constructed by iterated real blow up along corners of $[0,+\infty)^{\vec{\Gamma}_{1}}$. We collect basic properties of the partially compactified Schwinger spaces in the Appendix \ref{Schwinger spaces}. We use $\widetilde{[0,+\infty)}^{\vec{\Gamma}_{1}}$ to denote the partially compactified Schwinger space of $\vec{\Gamma}$.
\begin{prop}
    The closure of $S^{+}((0,+\infty)^{\vec{\Gamma}_{1}})$ in $\widetilde{[0,+\infty)}^{\vec{\Gamma}_{1}}$, which we denote by 
    $$
    \bar{S}^{+}((0,+\infty)^{\vec{\Gamma}_{1}}),
    $$
    is compact.
\end{prop}
%\begin{proof}
 %   This follows from the construction of $\widetilde{[0,+\infty)}^{\vec{\Gamma}_{1}}$.
%\end{proof}

Another important result that we will use pertaining to this compactification $\widetilde{[0,+\infty)}^{\vec{\Gamma}
_{1}}$ is the following.

\begin{lem}
    Let $\vec{\Gamma}$ be a connected directed graph, then the graphic Green's function
    $$
    d^{-1}_{\vec{\Gamma}}(t)_{ei}, \quad i\in \vec{\Gamma}_{0}-\{n\},\quad e\in \vec{\Gamma}_{1}
    $$
    can be extended to a smooth function on $\widetilde{[0,+\infty)}^{\vec{\Gamma}_{1}}$.
\end{lem}
%\begin{proof}
%    See Lemma \ref{extended functions}.
%\end{proof}

As a consequence, we have
\begin{cor}
    Let $(\vec{\Gamma},m,l)$ be a Feynman graph, such that $\vec{\Gamma}_{0}=\vec{I}$. Given 
    $$
    \beta\in (\omega_{\mathbb{A}^{d}}[d])^{\boxtimes\vec{I}}[(\mathbb{A}^{d})^{\vec{I}}]
    $$ 
    and
    $$
    e^{\sum_{i=1}^{n}z_{i}\cdot\lambda_{i}}=e^{\sum_{s=1}^{d}\sum_{i=1}^{n}z^{s}_{i}\lambda^{s}_{i}}\in C^{\infty}((\mathbb{A}^{d})^{\vec{I}}\times (\mathbb{A}^{d})^{\vec{I}}),
    $$
    the Feynman graph integrand
    $$
    W((\vec{\Gamma},m,l),\beta e^{\sum_{i=1}^{n}z_{i}\cdot\lambda_{i}})
    $$
    can be extended to a smooth differential form on $\widetilde{[0,+\infty)}^{\vec{\Gamma}_{1}}$. In particular,
    $$
    \int_{S^{+}((0,+\infty)^{\vec{\Gamma}_{1}})}W((\vec{\Gamma},m,l),\beta e^{\sum_{i=1}^{n}z_{i}\cdot\lambda_{i}})
    $$
    is convergent.
\end{cor}
\begin{proof}
    By Proposition \ref{explicit formula}, we have an explicit formula for $
    W((\vec{\Gamma},m,l),\beta e^{\sum_{i=1}^{n}z_{i}\cdot\lambda_{i}})
    $. Since $d^{-1}_{\vec{\Gamma}}(t)_{ei}$ is smooth on $\widetilde{[0,+\infty)}^{\vec{\Gamma}_{1}}$, the operator $d\hat{y}_{e}^{s}$ is also smooth. So $
    W((\vec{\Gamma},m,l),\beta e^{\sum_{i=1}^{n}z_{i}\cdot\lambda_{i}})
    $ can be extended to a smooth differential form. Finally, since $\bar{S}^{+}((0,+\infty)^{\vec{\Gamma}_{1}})$ is compact, the integral
    $$
    \int_{S^{+}((0,+\infty)^{\vec{\Gamma}_{1}})}W((\vec{\Gamma},m,l),\beta e^{\sum_{i=1}^{n}z_{i}\cdot\lambda_{i}})=
    \int_{\bar{S}^{+}((0,+\infty)^{\vec{\Gamma}_{1}})}W((\vec{\Gamma},m,l),\beta e^{\sum_{i=1}^{n}z_{i}\cdot\lambda_{i}})
    $$
    is convergent.
\end{proof}

Before we define residues on $\mathbf{J}_{\vec{I}}^{\mathbb{A}^{d}}((\omega_{\mathbb{A}^{d}}[d])^{\boxtimes\vec{I}})$, we state the following useful proposition:
\begin{prop}\label{useful property}
    Let $(\vec{\Gamma},m,l)$ be a Feynman graph, such that $\vec{\Gamma}_{0}=\vec{I}$. Given 
    $$
    \beta\in \omega_{\mathbb{A}^{d}}[d])^{\boxtimes\vec{I}}[(\mathbb{A}^{d})^{\vec{I}}]
    $$
    and
    $$
    e^{\sum_{i=1}^{n}z_{i}\cdot\lambda_{i}}=e^{\sum_{s=1}^{d}\sum_{i=1}^{n}z^{s}_{i}\lambda^{s}_{i}}\in C^{\infty}((\mathbb{A}^{d})^{\vec{I}}\times (\mathbb{A}^{d})^{\vec{I}}),
    $$
    then we have the following consequences:
    \begin{enumerate}
        \item The Feynman graph integrand
        $$
        W((\vec{\Gamma},m,l),\beta e^{\sum_{i=1}^{n}z_{i}\cdot\lambda_{i}})
        $$
        induces a well-defined differential form on $(0,+\infty)^{\vec{\Gamma}_{1}}/\mathbb{R}^{+}$. Here the $\mathbb{R}^{+}$ action is given by
        $$
        \lambda\cdot(t_{e_{1}},\dots,t_{e_{|\Gamma_{1}|}})=(\lambda\cdot t_{e_{1}},\dots,\lambda\cdot t_{e_{|\Gamma_{1}|}}),
        $$
        where $\lambda\in \mathbb{R}^{+}$, $(t_{e_{1}},\dots,t_{e_{|\Gamma_{1}|}})\in (0,+\infty)^{\vec{\Gamma}_{1}}$.
        \item Let $S\subset(0,+\infty)^{\vec{\Gamma}_{1}}$ be a submanifold, such that the natural map $S\rightarrow(0,+\infty)^{\vec{\Gamma}_{1}}/\mathbb{R}^{+}$ is a diffeomorphism. Then we have
        $$
        \int_{S}W((\vec{\Gamma},m,l),\beta e^{\sum_{i=1}^{n}z_{i}\cdot\lambda_{i}})=
        \int_{S^{+}((0,+\infty)^{\vec{\Gamma}_{1}})}W((\vec{\Gamma},m,l),\beta e^{\sum_{i=1}^{n}z_{i}\cdot\lambda_{i}}).
        $$
    \end{enumerate}
\end{prop}
\begin{proof}
    (2) is a direct consequence of (1). Let's prove (1).

    Let $\vec{V}=\vec{V}_{1}+\vec{V}_{2}$, where $\vec{V}_{1}=\sum_{s=1}^{d}\sum_{i=1}^{n-1}\bar{\tilde{z}}_{i}^{s}\partial_{\bar{\tilde{z}}_{i}^{s}}$, $\vec{V}_{2}=\sum_{e\in\vec{\Gamma}_{1}}t_{e}\partial_{t_{e}}$. To simplify notations, we define
    $$
    \tilde{W}=p_{(\vec{\Gamma},m,l)}(-e^{-z_{e}\cdot y_{e}}dy_{e}^{m(e)},-e^{-z_{e}\cdot y_{e}}(z_{e}\cdot dy_{e})dy_{e}^{m(e)})e^{\sum_{i=1}^{n}z_{i}\cdot\lambda_{i}}\otimes \beta.
    $$
    By Proposition \ref{Lie and interior product}, we have
    $$
    \begin{cases}
        L_{\vec{V}}\tilde{W}=0,\\
        \iota_{\vec{V}}\tilde{W}=0.
    \end{cases}
    $$
    So we can get 
    \begin{align*}
        L_{\vec{V}_{2}}W((\vec{\Gamma},m,l),\beta e^{\sum_{i=1}^{n}z_{i}\cdot\lambda_{i}})=\int_{(\mathbb{A}^{d})^{\vec{I}-\{n\}}}L_{\vec{V}_{2}}\tilde{W}=-\int_{(\mathbb{A}^{d})^{\vec{I}-\{n\}}}L_{\vec{V}_{1}}\tilde{W}=0.
    \end{align*}
    Similarly,
    $$
    \iota_{\vec{V}_{2}}W((\vec{\Gamma},m,l),\beta e^{\sum_{i=1}^{n}z_{i}\cdot\lambda_{i}})=\pm\int_{(\mathbb{A}^{d})^{\vec{I}-\{n\}}}\iota_{\vec{V}_{1}}\tilde{W}=0.
    $$
    We notice that $\vec{V}_{2}$ is the infinitesimal generator of $\mathbb{R}^{+}$ action, so $W((\vec{\Gamma},m,l),\beta e^{\sum_{i=1}^{n}z_{i}\cdot\lambda_{i}})$ induces a well-defined differential form on $(0,+\infty)^{\vec{\Gamma}_{1}}/\mathbb{R}^{+}$.
\end{proof}

Now, we study the dependency of $\int_{S^{+}((0,+\infty)^{\vec{\Gamma}_{1}})}W((\vec{\Gamma},m,l),\beta e^{\sum_{i=1}^{n}z_{i}\cdot\lambda_{i}})$ on the Feynman graph $(\vec{\Gamma},m,l)$. We have the following result:
\begin{prop}\label{well-defineness}
        Let $(\vec{\Gamma},m,l)$ be a Feynman graph, such that $\vec{\Gamma}_{0}=\vec{I}$. Given 
    $$
    \beta\in \omega_{\mathbb{A}^{d}}[d])^{\boxtimes\vec{I}}[(\mathbb{A}^{d})^{\vec{I}}]
    $$
    and
    $$
    e^{\sum_{i=1}^{n}z_{i}\cdot\lambda_{i}}=e^{\sum_{s=1}^{d}\sum_{i=1}^{n}z^{s}_{i}\lambda^{s}_{i}}\in C^{\infty}((\mathbb{A}^{d})^{\vec{I}}\times (\mathbb{A}^{d})^{\vec{I}}),
    $$
    then we have the following consequences:
    \begin{enumerate}
        \item Let $1\leq l_{1},l_{2}< l$, $l\leq l'_{1},l'_{2}\leq |\Gamma_{1}|$, we use $\sigma_{l_{1}l_{2}}(\vec{\Gamma})$ ($\sigma_{l'_{1}l'_{2}}(\vec{\Gamma})$) to denote the Feynman graph $\vec{\Gamma}$ with the order of $e_{l_{1}},e_{l_{2}}\in\vec{\Gamma}_{1}$ ($e_{l'_{1}},e_{l'_{2}}\in\vec{\Gamma}_{1}$) interchanged. Then we have
        $$
        \begin{cases}
            \int_{S^{+}((0,+\infty)^{\sigma_{l_{1}l_{2}}(\vec{\Gamma}_{1})}}W((\sigma_{l_{1}l_{2}}(\vec{\Gamma}),m,l),\beta e^{\sum_{i=1}^{n}z_{i}\cdot\lambda_{i}})=
        \int_{S^{+}((0,+\infty)^{\vec{\Gamma}_{1}})}W((\vec{\Gamma},m,l),\beta e^{\sum_{i=1}^{n}z_{i}\cdot\lambda_{i}})\\
        \int_{S^{+}((0,+\infty)^{\sigma_{l'_{1}l'_{2}}(\vec{\Gamma}_{1})}}W((\sigma_{l'_{1}l'_{2}}(\vec{\Gamma}),m,l),\beta e^{\sum_{i=1}^{n}z_{i}\cdot\lambda_{i}})=
        -\int_{S^{+}((0,+\infty)^{\vec{\Gamma}_{1}})}W((\vec{\Gamma},m,l),\beta e^{\sum_{i=1}^{n}z_{i}\cdot\lambda_{i}})
        \end{cases}
        $$
        \item For $1\leq s\leq d$, $i,j\in \vec{\Gamma}_{0}$, let $(\vec{\Gamma}',m^{s},l)$ be the Feynman graph which satisfies the following:
        \begin{itemize}
            \item $\vec{\Gamma}\subset\vec{\Gamma}'$ is a directed subgraph.
            \item $\vec{\Gamma}'_{0}=\vec{\Gamma}_{0}$.
            \item $\vec{\Gamma}'_{1}=\{e_{0}\}\cup\vec{\Gamma}_{1}$, where $t(e_{0})=i$, $h(e_{0})=j$.
            \item $m(e_{0})=s$.
        \end{itemize}
        Then we have
        \begin{align}\label{well-defined formula}
        \sum_{s=1}^{d}\int_{S^{+}((0,+\infty)^{\vec{\Gamma}'_{1}})}W((\vec{\Gamma}',m^{s},l),z_{ij}^{s}\beta e^{\sum_{i=1}^{n}z_{i}\cdot\lambda_{i}})=
        \int_{S^{+}((0,+\infty)^{\vec{\Gamma}_{1}})}W((\vec{\Gamma},m,l),\beta e^{\sum_{i=1}^{n}z_{i}\cdot\lambda_{i}}).    
        \end{align}
        \item For $1\leq s\leq d$, $i,j\in \vec{\Gamma}_{0}$, let $(\vec{\Gamma}'',m^{s},l)$ be the Feynman graph which satisfies the following:
        \begin{itemize}
            \item $\vec{\Gamma}\subset\vec{\Gamma}''$ is a directed subgraph.
            \item $\vec{\Gamma}''_{0}=\vec{\Gamma}_{0}$.
            \item $\vec{\Gamma}''_{1}=\vec{\Gamma}_{1}\cup\{e_{|\vec{\Gamma}_{1}|+1}\}$, where $t(e_{|\vec{\Gamma}_{1}|+1})=i$, $h(e_{|\vec{\Gamma}_{1}|+1})=j$.
            \item $m(e_{|\vec{\Gamma}_{1}|+1})=s$.
        \end{itemize}
        Then we have
        $$
        \sum_{s=1}^{d}\int_{S^{+}((0,+\infty)^{\vec{\Gamma}''_{1}})}W((\vec{\Gamma}'',m^{s},l),z_{ij}^{s}\beta e^{\sum_{i=1}^{n}z_{i}\cdot\lambda_{i}})=0.
        $$
    \end{enumerate}
\end{prop}
\begin{proof}
    (1) can be verified directly. Let's prove (2). Let's first assume $\vec{\Gamma}_{1}\neq\emptyset$. To simplify notations, we define
    $$
    \tilde{W}=p_{(\vec{\Gamma},m,l)}(-e^{-z_{e}\cdot y_{e}}dy_{e}^{m(e)},-e^{-z_{e}\cdot y_{e}}(z_{e}\cdot dy_{e})dy_{e}^{m(e)})e^{\sum_{i=1}^{n}z_{i}\cdot\lambda_{i}}\otimes \beta.
    $$
    Let's consider a submanifold
    $$
    S=S^{+}((0,+\infty)^{\vec{\Gamma}_{1}})\times(0,+\infty)\subset S^{+}((0,+\infty)^{\vec{\Gamma}'_{1}}).
    $$
    We notice that the natural map from $S$ to $(0,+\infty)^{\vec{\Gamma}'_{1}}/\mathbb{R}^{+}$ is a diffeomophism, so
    \begin{align*}
        &\sum_{s=1}^{d}\int_{S^{+}((0,+\infty)^{\vec{\Gamma}'_{1}})}W((\vec{\Gamma}',m^{s},l),z_{ij}^{s}\beta e^{\sum_{i=1}^{n}z_{i}\cdot\lambda_{i}})\\
        &=
        \sum_{s=1}^{d}\int_{S^{+}((0,+\infty)^{\vec{\Gamma}_{1}})}\int_{(0,+\infty)}W((\vec{\Gamma}',m^{s},l),z_{ij}^{s}\beta e^{\sum_{i=1}^{n}z_{i}\cdot\lambda_{i}})\\
        &=\sum_{s=1}^{d}\int_{S^{+}((0,+\infty)^{\vec{\Gamma}_{1}})}\int_{(\mathbb{A}^{d})^{\vec{I}-\{n\}}}z_{ij}^{s}x_{ij}^{s}\tilde{W}\\
        &=
        \int_{S^{+}((0,+\infty)^{\vec{\Gamma}_{1}})}W((\vec{\Gamma},m,l),\beta e^{\sum_{i=1}^{n}z_{i}\cdot\lambda_{i}}).
    \end{align*}
    When $\vec{\Gamma}_{1}=\emptyset$, $|\Gamma_{0}|\geq3$, $\vec{\Gamma}$ and $\vec{\Gamma}'$ are disconnected, so both sides of formula $(\ref{well-defined formula})$ is zero. When $\vec{\Gamma}_{1}=\emptyset$, $i=1$, $j=2$, $\vec{\Gamma}_{0}=\{1<2\}$, we have
    \begin{align*}
        &\sum_{s=1}^{d}\int_{S^{+}((0,+\infty)^{\vec{\Gamma}'_{1}})}W((\vec{\Gamma}',m^{s},l),z_{ij}^{s}\beta e^{\sum_{i=1}^{n}z_{i}\cdot\lambda_{i}})\\
        &=
        \left.\left(
        -\int_{\mathbb{A}^{d}}e^{-z_{12}\cdot y_{12}}(z_{12}\cdot dy_{12})\beta e^{z_{1}\cdot \lambda_{1}+z_{2}\cdot \lambda_{2}}
        \right)\right|_{t_{12}=1}\\
        &=
        \left.\left(
        \int_{\mathbb{A}^{d}}\bar{\partial}\left(e^{-z_{12}\cdot y_{12}}\beta e^{z_{1}\cdot \lambda_{1}+z_{2}\cdot \lambda_{2}}
        \right)\right)\right|_{t_{12}=1}\\
        &=0.
    \end{align*}
    Since $\vec{\Gamma}$ is disconnected, 
    \begin{align*}
        &\int_{S^{+}((0,+\infty)^{\vec{\Gamma}_{1}})}W((\vec{\Gamma},m,l),\beta e^{\sum_{i=1}^{n}z_{i}\cdot\lambda_{i}})\\
        &=0\\
        &=
        \sum_{s=1}^{d}\int_{S^{+}((0,+\infty)^{\vec{\Gamma}'_{1}})}W((\vec{\Gamma}',m^{s},l),z_{ij}^{s}\beta e^{\sum_{i=1}^{n}z_{i}\cdot\lambda_{i}}).
    \end{align*}
    (3) can be proved by similar arguments.
\end{proof}

Now, we can define residues on $\mathbf{J}_{\vec{I}}^{\mathbb{A}^{d}}((\omega_{\mathbb{A}^{d}}[d])^{\boxtimes\vec{I}})$.

\begin{defn}
    Let
    $$
    \alpha=p(x_{ij}^{s},\mathbf{d}x_{ij}^{s})\otimes\beta\in \mathbf{J}_{\vec{I}}^{\mathbb{A}^{d}}((\omega_{\mathbb{A}^{d}}[d])^{\boxtimes\vec{I}}),
    $$
    where $p(x_{ij}^{s},\mathbf{d}x_{ij}^{s})$ is a monomial with coefficient equals $1$. Let
    $$
    e^{\sum_{i=1}^{n}z_{i}\cdot\lambda_{i}}=e^{\sum_{s=1}^{d}\sum_{i=1}^{n}z^{s}_{i}\lambda^{s}_{i}}\in C^{\infty}((\mathbb{A}^{d})^{\vec{I}}\times (\mathbb{A}^{d})^{\vec{I}}),
    $$
    then the residue of $\alpha e^{\sum_{i=1}^{n}z_{i}\cdot\lambda_{i}}$ is defined by
    \begin{align*}
        &\frac{-1}{(-2\pi i)^{d(n-1)}}\oint_{z_{1},\dots z_{n-1}=z_{n}}\alpha e^{\sum_{i=1}^{n}z_{i}\cdot\lambda_{i}}\\
        &=\frac{(-1)^{\frac{1}{2}(|\Gamma_{1}|-l)(\Gamma_{1}|-l+1)+1}}{(-2\pi i)^{d(n-1)}}\int_{S^{+}((0,+\infty)^{\vec{\Gamma}_{1}})}W((\vec{\Gamma},m,l),\beta e^{\sum_{i=1}^{n}z_{i}\cdot\lambda_{i}}),
    \end{align*}
    where $(\vec{\Gamma},m,l)$ is a Feynman graph whose corresponding polynomial $p_{(\vec{\Gamma},m,l)}=p(x_{ij}^{s},\mathbf{d}x_{ij}^{s})$. We can define residues for general elements in $\mathbf{J}_{\vec{I}}^{\mathbb{A}^{d}}((\omega_{\mathbb{A}^{d}}[d])^{\boxtimes\vec{I}})$ by linear extension. By Proposition \ref{well-defineness}, residues are well-defined.
\end{defn}

Finally, let's prove a useful property for residues on $\mathbf{J}_{\vec{I}}^{\mathbb{A}^{d}}((\omega_{\mathbb{A}^{d}}[d])^{\boxtimes\vec{I}})$.
\begin{prop}\label{residue 0}
    Let $\alpha\in \mathbf{J}_{\vec{I}}^{\mathbb{A}^{d}}((\omega_{\mathbb{A}^{d}}[d])^{\boxtimes\vec{I}})$, $e^{\sum_{i=1}^{n}z_{i}\cdot\lambda_{i}}\in C^{\infty}((\mathbb{A}^{d})^{\vec{I}}\times (\mathbb{A}^{d})^{\vec{I}})$, and $1\leq s\leq d$. We have
    
    \begin{enumerate}
        \item If $i\in\vec{I}-\{n\}$, then
        $$
        \frac{-1}{(-2\pi i)^{d(n-1)}}\oint_{z_{1},\dots z_{n-1}=z_{n}}\left(\alpha e^{\sum_{i=1}^{n}z_{i}\cdot\lambda_{i}}\right)\cdot \partial_{z_{i}^{s}}=0.
        $$
        \item If $i=n$, then
        \begin{align*}
            &\frac{-1}{(-2\pi i)^{d(n-1)}}\oint_{z_{1},\dots z_{n-1}=z_{n}}\left(\alpha e^{\sum_{i=1}^{n}z_{i}\cdot\lambda_{i}}\right)\cdot \partial_{z_{i}^{s}}\\
            &=\left(\frac{-1}{(-2\pi i)^{d(n-1)}}\oint_{z_{1},\dots z_{n-1}=z_{n}}\alpha e^{\sum_{i=1}^{n}z_{i}\cdot\lambda_{i}}\right)\cdot \partial_{z_{n}^{s}}.
        \end{align*}
    \end{enumerate}
    \begin{proof}
        Without loss of generality, we assume $\alpha=p_{(\vec{\Gamma},m,l)}\otimes\beta$, where $(\vec{\Gamma},m,l)$ is a Feynman graph. We can also assume $\vec{\Gamma}_{1}\neq\emptyset$, $t(e_{1})=i$. 
        
        Let $\vec{\Gamma}'\subset\vec{\Gamma}$ be a subgraph, such that $\vec{\Gamma}_{0}'=\vec{\Gamma}_{0}$, $\vec{\Gamma}_{1}'=\vec{\Gamma}_{1}-\{e_{1}\}$. Let
        $$
        \tilde{W}'=p_{(\vec{\Gamma}',m|_{\vec{\Gamma}'},l)}(-e^{-z_{e}\cdot y_{e}}dy_{e}^{m(e)},-e^{-z_{e}\cdot y_{e}}(z_{e}\cdot dy_{e})dy_{e}^{m(e)})e^{\sum_{i=1}^{n}z_{i}\cdot\lambda_{i}}\otimes \beta.
        $$
        When $1<l$, we have 
        $$
        \partial_{z_{i}^{s}}x_{e_{1}}^{m(e_{1})}=-x_{e_{1}}^{s}x_{e_{1}}^{m(e_{1})}.
        $$
        Let's prove
        \begin{align*}
            &\int_{S^{+}((0,+\infty)^{|\Gamma_{_{1}}|+1})}\int_{(\mathbb{A}^{d})^{\vec{I}-\{n\}}}
            e^{-\frac{z_{e_{1}}\cdot \bar{z}_{e_{1}}}{t_{e_{0}}}}d(\frac{\bar{z}_{e_{1}}^{s}}{t_{e_{0}}})e^{-\frac{z_{e_{1}}\cdot \bar{z}_{e_{1}}}{t_{e_{1}}}}d(\frac{\bar{z}_{e_{1}}^{m(e_{1})}}{t_{e_{1}}})\tilde{W}'\\
            &=
            -\int_{S^{+}((0,+\infty)^{\vec{\Gamma}_{1}})}\int_{(\mathbb{A}^{d})^{\vec{I}-\{n\}}}\partial_{z_{i}^{s}}\left(e^{-\frac{z_{e_{1}}\cdot \bar{z}_{e_{1}}}{t_{e_{1}}}}d(\frac{\bar{z}_{e_{1}}^{m(e_{1})}}{t_{e_{1}}})\right)\tilde{W}'.
        \end{align*}
        In other words, the formal $\mathcal{D}_{(\mathbb{A}^{d})^{\vec{I}}}$-module structure on $\mathbf{J}_{\vec{I}}^{\mathbb{A}^{d}}$ coincide with the actual derivative when we compute residues. 

        When $\vec{\Gamma}_{1}'\neq\emptyset$, let
        $$
        S=S^{+}((0,+\infty)^{\vec{\Gamma}_{1}'})\times(0,+\infty)^{2}\subset(0,+\infty)^{|\Gamma_{1}|+1}.
        $$
        Note the natural map from $S$ to $(0,+\infty)^{|\Gamma_{1}|+1}/\mathbb{R}^{+}$ is a diffeomorphism, we have
        \begin{align*}
            &\int_{S^{+}((0,+\infty)^{|\Gamma_{_{1}}|+1})}\int_{(\mathbb{A}^{d})^{\vec{I}-\{n\}}}
            e^{-\frac{z_{e_{1}}\cdot \bar{z}_{e_{1}}}{t_{e_{0}}}}d(\frac{\bar{z}_{e_{1}}^{s}}{t_{e_{0}}})e^{-\frac{z_{e_{1}}\cdot \bar{z}_{e_{1}}}{t_{e_{1}}}}d(\frac{\bar{z}_{e_{1}}^{m(e_{1})}}{t_{e_{1}}})\tilde{W}'\\
            &=
            \int_{S^{+}((0,+\infty)^{\vec{\Gamma}'_{_{1}}})}\int_{(\mathbb{A}^{d})^{\vec{I}-\{n\}}}i(x_{e_{1}}^{s})i(x_{e_{1}}^{m(e_{1})})\tilde{W}'\\
            &=
            -\int_{S^{+}((0,+\infty)^{\vec{\Gamma}'_{_{1}}})}\int_{(\mathbb{A}^{d})^{\vec{I}-\{n\}}}\partial_{z_{i}^{s}}(i(x_{e_{1}}^{m(e_{1})}))\tilde{W}'\\
            &=
            -\int_{S^{+}((0,+\infty)^{\vec{\Gamma}_{1}})}\int_{(\mathbb{A}^{d})^{\vec{I}-\{n\}}}\partial_{z_{i}^{s}}\left(e^{-\frac{z_{e_{1}}\cdot \bar{z}_{e_{1}}}{t_{e_{1}}}}d(\frac{\bar{z}_{e_{1}}^{m(e_{1})}}{t_{e_{1}}})\right)\tilde{W}'.
        \end{align*}
        When $\vec{\Gamma}_{1}'=\emptyset$, this can be proved by directed computation.

        We can use similar arguments to prove the case when $l=1$.

        Now, (1) follows from the fact that the integral of a total derivative is zero. By dominated convergence theorem for derivatives, we can interchange the order of taking derivative and integration. So we proved (2).
    \end{proof}
\end{prop}
\subsection{Construction of the unit chiral algebra on $\omega_{\mathbb{A}^{d}}[d-1]$}.

In this subsection, we will construct the $L_{\infty}$ chiral algebra structure on $\omega_{\mathbb{A}^{d}}[d-1]$ by using residues.

We first introduce the following definition:
\begin{defn}
    Let $\alpha\in \mathbf{J}_{\vec{I}}^{\mathbb{A}^{d}}((\omega_{\mathbb{A}^{d}}[d])^{\boxtimes\vec{I}})$, the shifted $n$ operation
    $$
    \tilde{\mu}_{\vec{I}}:\mathbf{J}_{\vec{I}}^{\mathbb{A}^{d}}((\omega_{\mathbb{A}^{d}}[d])^{\boxtimes\vec{I}})\rightarrow \omega_{\mathbb{A}^{d}}[d][\mathbb{A}^{d}]\otimes_{k[\lambda_{\bullet}]}k[\lambda_{1},\dots,\lambda_{n}]
    $$
    is given by
    $$
    \tilde{\mu}_{\vec{I}}(\alpha)=\frac{-e^{-\lambda_{\bullet}\cdot w}}{(-2\pi i)^{d(n-1)}}\left.\left(\oint_{z_{1},\cdots z_{n-1}=z_{n}}\alpha e^{\sum_{j=1}^{n}z_{j}\cdot \lambda_{j}}\right)\right|_{z_n=w},
    $$
    where $\lambda_{\bullet}=\sum_{i=1}^{n}\lambda_{i}$.
\end{defn}
\begin{rem}
    By Proposition \ref{explicit formula}, we know $\tilde{\mu}_{\vec{I}}(\alpha)$ is indeed an element of $\omega_{\mathbb{A}^{d}}[d][\mathbb{A}^{d}]\otimes_{k[\lambda_{\bullet}]}k[\lambda_{1},\dots,\lambda_{n}]$, i.e., $\tilde{\mu}_{\vec{I}}(\alpha)$ is a polynomial with respect to $\{\lambda_{i}\}_{i\in\vec{I}}$.
\end{rem}
\begin{prop}
    $\tilde{\mu}_{\vec{I}}(\alpha)$ is a homological degree $1$ element in $\mathrm{Hom}^{ch}((\omega_{\mathbb{A}^{d}}[d])^{\boxtimes\vec{I}},\omega_{\mathbb{A}^{d}}[d])$.
\end{prop}
\begin{proof}
    We first prove $\tilde{\mu}$ is of homological degree $1$. Assume
    $$
    \alpha=p\otimes\beta\in\mathbf{J}_{\vec{I}}^{\mathbb{A}^{d}}\otimes_{\mathcal{O}_{(\mathbb{A}^{d})^{\vec{I}}}[(\mathbb{A}^{d})^{\vec{I}}]}(\omega_{\mathbb{A}^{d}}[d])^{\boxtimes\vec{I}}[(\mathbb{A}^{d})^{\vec{I}}],
    $$
    we use $|p|$ and $|\beta|$ to denote the homological degree of $p$ and $\beta$ respectively. To make sure the integral
    $$
    \oint_{z_{1},\cdots z_{n-1}=z_{n}}p\otimes\beta e^{\sum_{j=1}^{n}z_{j}\cdot \lambda_{j}}
    $$
    is non-zero, we have
    $$
    \begin{cases}
        |\Gamma_{1}|+|p|=d(|\Gamma_{0}|-1)+|\Gamma_{1}|-1\\
        |\beta|=-d|\Gamma_{0}|
    \end{cases}.
    $$
    So $|\tilde{\mu}_{\vec{I}}(\alpha)|=-d=|p|+|\beta|+1=|\alpha|+1$.

    Now, let's prove $\tilde{\mu}_{\vec{I}}$ is a $\mathcal{D}_{(\mathbb{A}^{d})^{\vec{I}}}$-module map. This follows from the following computations:
    \begin{align*}
        &\tilde{\mu}_{\vec{I}}(\alpha\cdot z_{i}^{s})\\
        &=
        \frac{-e^{-\lambda_{\bullet}\cdot w}}{(-2\pi i)^{d(n-1)}}\left.\left(\oint_{z_{1},\cdots z_{n-1}=z_{n}}z_{i}^{s}\alpha e^{\sum_{j=1}^{n}z_{j}\cdot \lambda_{j}}\right)\right|_{z_n=w}\\
        &=
        \frac{-e^{-\lambda_{\bullet}\cdot w}}{(-2\pi i)^{d(n-1)}}\left.\left(\oint_{z_{1},\cdots z_{n-1}=z_{n}}\alpha \partial_{\lambda_{i}^{s}}e^{\sum_{j=1}^{n}z_{j}\cdot \lambda_{j}}\right)\right|_{z_n=w}\\
        &=(w+\partial_{\lambda_{i}^{s}})\frac{-e^{-\lambda_{\bullet}\cdot w}}{(-2\pi i)^{d(n-1)}}\left.\left(\oint_{z_{1},\cdots z_{n-1}=z_{n}}\alpha e^{\sum_{j=1}^{n}z_{j}\cdot \lambda_{j}}\right)\right|_{z_n=w}\\
        &=
        \tilde{\mu}_{\vec{I}}(\alpha)\cdot z_{i}^{s},
    \end{align*}
    \begin{align*}
        &\tilde{\mu}_{\vec{I}}(\alpha\cdot \partial_{z_{i}^{s}})\\
        &=
        \frac{e^{-\lambda_{\bullet}\cdot w}}{(-2\pi i)^{d(n-1)}}\left.\left(\oint_{z_{1},\cdots z_{n-1}=z_{n}}\partial_{z_{i}^{s}}(\alpha) e^{\sum_{j=1}^{n}z_{j}\cdot \lambda_{j}}\right)\right|_{z_n=w}\\
        &=
        \frac{e^{-\lambda_{\bullet}\cdot w}}{(-2\pi i)^{d(n-1)}}\left.\left(\oint_{z_{1},\cdots z_{n-1}=z_{n}}\partial_{z_{i}^{s}}(\alpha e^{\sum_{j=1}^{n}z_{j}\cdot \lambda_{j}})\right)\right|_{z_n=w}\\
        &-
        \frac{e^{-\lambda_{\bullet}\cdot w}}{(-2\pi i)^{d(n-1)}}\left.\left(\oint_{z_{1},\cdots z_{n-1}=z_{n}}\lambda_{i}^{s}\alpha e^{\sum_{j=1}^{n}z_{j}\cdot \lambda_{j}}\right)\right|_{z_n=w}\\
        &=
        \frac{e^{-\lambda_{\bullet}\cdot w}}{(-2\pi i)^{d(n-1)}}\left.\left(\oint_{z_{1},\cdots z_{n-1}=z_{n}}\partial_{z_{i}^{s}}(\alpha e^{\sum_{j=1}^{n}z_{j}\cdot \lambda_{j}})\right)\right|_{z_n=w}+
        \tilde{\mu}_{\vec{I}}(\alpha)\cdot \partial_{\lambda_{i}^{s}}.
    \end{align*}
    Using Proposition \ref{residue 0}, the first term is $0$ when $i\neq n$. When $i=n$, the first term is
    $$
    (-\lambda_{\bullet}+\partial_{w})\frac{e^{-\lambda_{\bullet}\cdot w}}{(-2\pi i)^{d(n-1)}}\left.\left(\oint_{z_{1},\cdots z_{n-1}=z_{n}}\alpha e^{\sum_{j=1}^{n}z_{j}\cdot \lambda_{j}}\right)\right|_{z_n=w},
    $$
    which is zero as an element in
    $$
    \omega_{\mathbb{A}^{d}}[d][\mathbb{A}^{d}]\otimes_{k[\lambda_{\bullet}]}k[\lambda_{1},\dots,\lambda_{n}].
    $$
\end{proof}

Let's prove $\tilde{\mu}_{\vec{I}}$ satisfy the following shifted $L_{\infty}$ relations:
\begin{thm}
    $\{\tilde{\mu}_{\vec{I}}\}_{\vec{I}\in \overrightarrow{\mathbf{fSet}}}$ satisfies the following properties:
    \begin{enumerate}
        \item $\tilde{\mu}_{\vec{I}}=\tilde{\mu}_{\vec{I}_{\sigma_{ii'}}}\circ\tau_{\sigma_{ii'}}$, where $\sigma_{ii'}$ is the permutation of $i\neq i'\in\vec{I}$.
        \item 
        $$
        -\tilde{\mu}_{\vec{I}}\circ \mathbf{d}=\sum_{\vec{I}'\subset\vec{I}}\tilde{\mu}_{\{\bullet\}\cup\vec{I}-\vec{I}'}\circ \tilde{\mu}_{\vec{I}'\subset\vec{I}}
        $$
    \end{enumerate}
\end{thm}
\begin{proof}
    Let's prove (1) first. When $i,i'\neq n$, it is trivial. Let's assume $i<i'=n$. Since $\tilde{\mu}_{\vec{I}}$ is a $\mathcal{D}_{(\mathbb{A}^{d})^{\vec{I}}}$-module morphism, we only need to prove this in the case when 
    $$
    \alpha=p(x_{i}^{s},\mathbf{d}x_{i}^{s})\otimes\prod_{i=1}^{n}d^{d}z_{i}.
    $$
    In this case,
    $$
    \tilde{\mu}_{\vec{I}}(\alpha)=\frac{-e^{-\lambda_{\bullet}\cdot w}}{(-2\pi i)^{d(n-1)}}\left.\left(\oint_{z_{1},\cdots z_{n-1}=z_{n}}\alpha e^{\sum_{j=1}^{n}z_{j}\cdot \lambda_{j}}\right)\right|_{z_{n}=w}.
    $$
    By using the coordinate transform
    $$
    z_{j}=z'_{j}+2w-z'_{n}-z'_{i},
    $$
    we have 
    $$
    \tilde{\mu}_{\vec{I}}(\alpha)=\frac{-e^{-\lambda_{\bullet}\cdot w}}{(-2\pi i)^{d(n-1)}}\left.\left(\oint_{z'_{1},\cdots, z'_{i-1},z'_{n}, z'_{i+1},\cdots, z'_{n-1}=z'_{i}}(-1)^{d^2}\alpha e^{\sum_{j=1}^{n}z'_{j}\cdot \lambda_{j}+\lambda_{\bullet}\cdot(z'_i-z'_n)}\right)\right|_{z'_{i}=w}.
    $$
    By Proposition \ref{explicit formula} and the fact that $d^{d}w\cdot\lambda_{\bullet}^{s}=0$, we have
    $$
    \tilde{\mu}_{\vec{I}}(\alpha)=\frac{-e^{-\lambda_{\bullet}\cdot w}}{(-2\pi i)^{d(n-1)}}\left.\left(\oint_{z'_{1},\cdots, z'_{i-1},z'_{n}, z'_{i+1},\cdots, z'_{n-1}=z'_{i}}\tau_{\sigma_{in}}(\alpha) e^{\sum_{j=1}^{n}z'_{j}\cdot \lambda_{j}}\right)\right|_{z'_{i}=w}=\tilde{\mu}_{\vec{I}_{\sigma_{in}}}\circ\tau_{\sigma_{in}}(\alpha).
    $$

    Now, let's prove (2). Assume $\alpha=p_{(\vec{\Gamma},m,l)}\otimes\beta$, where $(\vec{\Gamma},m,l)$ is a Feynman graph. Let
    $$
    \tilde{W}=p_{(\vec{\Gamma},m,l)}(-e^{-z_{e}\cdot y_{e}}dy_{e}^{m(e)},-e^{-z_{e}\cdot y_{e}}(z_{e}\cdot dy_{e})dy_{e}^{m(e)})e^{\sum_{j=1}^{n}z_{j}\cdot\lambda_{j}}\otimes\beta.
    $$
    By Stokes' theorem and Proposition \ref{boundary description},
    \begin{align*}
        &-\tilde{\mu}_{\vec{I}}(\mathbf{d}\alpha)\\
        &=\frac{e^{-\lambda_{\bullet}\cdot w}(-1)^{\frac{1}{2}(|\Gamma_{1}|-l)(\Gamma_{1}|-l+1)}}{(-2\pi i)^{d(n-1)}}\left.\left(\int_{S^{+}((0,+\infty)^{\vec{\Gamma}_{1}})}\int_{(\mathbb{A}^{d})^{\vec{I}-\{n\}}}(\bar{\partial}+d_t)\tilde{W}\right)\right|_{z_n=w}\\
        &=
        \frac{e^{-\lambda_{\bullet}\cdot w}(-1)^{\frac{1}{2}(|\Gamma_{1}|-l)(\Gamma_{1}|-l+1)}}{(-2\pi i)^{d(n-1)}}\left.\left(\int_{S^{+}((0,+\infty)^{\vec{\Gamma}_{1}})}d_{t}\left(\int_{(\mathbb{A}^{d})^{\vec{I}-\{n\}}}\tilde{W}\right)\right)\right|_{z_n=w}\\
        &=
        \frac{e^{-\lambda_{\bullet}\cdot w}(-1)^{\frac{1}{2}(|\Gamma_{1}|-l)(\Gamma_{1}|-l+1)}}{(-2\pi i)^{d(n-1)}}\left(\sum_{\vec{\Gamma}'\subset \vec{\Gamma}}\mathrm{sgn}(\sigma_{\vec{\Gamma}'_{1}\subset \vec{\Gamma}_{1}})\int_{S^{+}((0,+\infty)^{\vec{\Gamma}_{1}- \vec{\Gamma}'_{1}})}\int_{S^{+}((0,+\infty)^{\vec{\Gamma}'_{1}})}\right.\\
        &
        \left.\left.\int_{(\mathbb{A}^{d})^{\{\bullet\}\cup\vec{\Gamma}_{0}-\vec{\Gamma}'_{0}-\{n\}}}\int_{(\mathbb{A}^{d})^{|\Gamma'_{0}|-1}}\tilde{W}\right)\right|_{z_n=w}\\
        &=
        \sum_{\vec{I}'\subset\vec{I}}\tilde{\mu}_{\{\bullet\}\cup\vec{I}-\vec{I}'}\circ \tilde{\mu}_{\vec{I}'\subset\vec{I}}(\alpha).
    \end{align*}
\end{proof}

We use the canonical décalage isomorphism
$$
\mathrm{Sym}^{\boxtimes I}(\omega_{\mathbb{A}^{d}}[d])\cong \wedge^{\boxtimes I}(\omega_{\mathbb{A}^{d}}[d-1])[n],
$$
the shifted $L_{\infty}$ chiral algebra structure $\{\tilde{\mu}_{\vec{I}}\}_{\vec{I}\in \overrightarrow{\mathbf{fSet}}}$ on $\omega_{\mathbb{A}^{d}}[d]$ corresponds to an $L_{\infty}$ chiral algebra structure on $\omega_{\mathbb{A}^{d}}[d-1]$.
\begin{defn}
    The $L_{\infty}$ algebra structure on $\omega_{\mathbb{A}^{d}}[d-1]$ is called the unit $L_{\infty}$ chiral algebra structure. We denote it by $\mathbf{Unit}[\mathbb{A}^d]$. The corresponding $L_{\infty}$ operations are denoted by $\{\mu^{\mathbf{Unit}}_{\vec{I}}\}_{\vec{I}\in \overrightarrow{\mathbf{fSet}}}$.
\end{defn}
\subsection{d=2}
In this subsection, we compute the 2-operation on a special element called propagator. We assume $d=2$ in this subsection.
\begin{defn}
    The propagator $P_{12}$ is an element of $\mathbf{J}_{\{1<2\}}^{\mathbb{A}^{2}}$, which is given by 
    $$
    P_{12}=x_{12}^1\mathbf{d}x_{12}^{2}-x_{12}^2\mathbf{d}x_{12}^{1}.
    $$
    \begin{prop}
        We have the following identity:
        $$
        \mu^{\mathbf{Unit}}_{\{1<2\}}(P_{12}d^{2}z_{1}\boxtimes d^{2}z_{2})=d^{2}w.
        $$
    \end{prop}
    \begin{proof}
    Note $P_{12}$ contains two terms. We first find the Feynman graphs which correspond to these two monomials. Let $\vec{\Gamma}$ be the directed graph constructed as follows:
        \begin{itemize}
            \item $\vec{\Gamma}_{0}=\{1<2\}$ and $\vec{\Gamma}_{1}=\{e_{1},e_{2}\}$.
            \item $t(e_{1})=t(e_{2})=1$ and $h(e_{1})=h(e_{2})=2$.
        \end{itemize}
        We consider two Feynman graphs $(\vec{\Gamma},m_{1},2)$ and $(\vec{\Gamma},m_{2},2)$, where $m_{1}$ and $m_{2}$ is defined by
        $$
        \begin{cases}
            m_{1}(e_{1})=1, m_{1}(e_{2})=2.\\
            m_{2}(e_{1})=2, m_{2}(e_{2})=1.
        \end{cases}
        $$
        Then we have $p_{(\vec{\Gamma},m_{1},2)}=x_{12}^1\mathbf{d}x_{12}^{2}$ and $p_{(\vec{\Gamma},m_{2},2)}=x_{12}^{2}\mathbf{d}x_{12}^{1}$. The graphic Green's function can be obtained by Corollary \ref{boundness}:
        $$
        \begin{cases}
            d^{-1}_{\vec{\Gamma}}(t)_{e_{1}1}=\frac{t_{e_{2}}}{t_{e_{1}}+t_{e_{2}}},\\
            d^{-1}_{\vec{\Gamma}}(t)_{e_{2}1}=\frac{t_{e_{1}}}{t_{e_{1}}+t_{e_{2}}}.
        \end{cases}
        $$
        
        Let's compute $\mu^{\mathbf{Unit}}_{\{1<2\}}(p_{(\vec{\Gamma},m_{1},2)}d^{2}z_{1}\boxtimes d^{2}z_{2})$. By Proposition \ref{explicit formula}, we have
        \begin{align*}
            &\mu^{\mathbf{Unit}}_{\{1<2\}}(p_{(\vec{\Gamma},m_{1},2)}d^{2}z_{1}\boxtimes d^{2}z_{2})\\
            &=\left(\int_{S^{+}((0,+\infty)^{\vec{\Gamma}_{1}})}
        \iota_{d^{2}\tilde{z}_{1}d^{2}\bar{\tilde{z}}_{1}}\right.\\
        &\left.\left.\circ (-d\hat{y}_{e_{1}}^{1})\circ(-(\sum_{s=1}^{2}(\partial_{\lambda_{e_{2}}^{s}}+\tilde{z}_{e_{2}}^{s})\circ d\hat{y}_{e_{2}}^{s})\circ d\hat{y}_{e_{2}}^{2})
        (d^{2}z_{1}\boxtimes d^{2}z_{2})\right)\right|_{\tilde{z}_{1}=0,\tilde{z}_{2}=w}\\
        &=
        \left(\int_{S^{+}((0,+\infty)^{\vec{\Gamma}_{1}})}\frac{t_{e_{1}}}{t_{e_{1}}+t_{e_{2}}}\left(\frac{t_{e_{1}}}{t_{e_{1}}+t_{e_{2}}}d(\frac{t_{e_{2}}}{t_{e_{1}}+t_{e_{2}}})-\frac{t_{e_{2}}}{t_{e_{1}}+t_{e_{2}}}d(\frac{t_{e_{1}}}{t_{e_{1}}+t_{e_{2}}})\right)\right)d^{2}w.
        \end{align*}
        Similarly, we have
        \begin{align*}
            &\mu^{\mathbf{Unit}}_{\{1<2\}}(-p_{(\vec{\Gamma},m_{2},2)}d^{2}z_{1}\boxtimes d^{2}z_{2})\\
            &=
        \left(\int_{S^{+}((0,+\infty)^{\vec{\Gamma}_{1}})}\frac{t_{e_{2}}}{t_{e_{1}}+t_{e_{2}}}\left(\frac{t_{e_{1}}}{t_{e_{1}}+t_{e_{2}}}d(\frac{t_{e_{2}}}{t_{e_{1}}+t_{e_{2}}})-\frac{t_{e_{2}}}{t_{e_{1}}+t_{e_{2}}}d(\frac{t_{e_{1}}}{t_{e_{1}}+t_{e_{2}}})\right)\right)d^{2}w.
        \end{align*}
        By Proposition \ref{useful property}, we can get
        \begin{align*}
            &\mu^{\mathbf{Unit}}_{\{1<2\}}(P_{12}d^{2}z_{1}\boxtimes d^{2}z_{2})\\
            &=
        \left(\int_{S^{+}((0,+\infty)^{\vec{\Gamma}_{1}})}\left(\frac{t_{e_{1}}}{t_{e_{1}}+t_{e_{2}}}d(\frac{t_{e_{2}}}{t_{e_{1}}+t_{e_{2}}})-\frac{t_{e_{2}}}{t_{e_{1}}+t_{e_{2}}}d(\frac{t_{e_{1}}}{t_{e_{1}}+t_{e_{2}}})\right)\right)d^{2}w\\
        &=\left(\int_{\{(t_{e_{1}},t_{e_{2}})\in (0,+\infty)^2|t_{e_{1}}+t_{e_{2}}=1\}}\left(\frac{t_{e_{1}}}{t_{e_{1}}+t_{e_{2}}}d(\frac{t_{e_{2}}}{t_{e_{1}}+t_{e_{2}}})-\frac{t_{e_{2}}}{t_{e_{1}}+t_{e_{2}}}d(\frac{t_{e_{1}}}{t_{e_{1}}+t_{e_{2}}})\right)\right)d^{2}w\\
        &=d^{2}w.
        \end{align*}
    \end{proof}
\end{defn}
\iffalse
\subsection{Trace map on the compacted supported chiral chain complex}

\begin{thm}
    There exist a trace map on the compacted supported unit chiral chain complex
    $$
    \mathbf{Tr}:C^{ch}_c(\mathbb{A}^d,\omega_{\mathbb{A}^d}[d-1])\rightarrow\mathbb{C}
    $$
    $$
    \mathbf{Tr}\left((\bar{\partial}+\mathbf{d}+d_{\mu_2}+d_{\mu_3}+\cdots)(-)\right)=0.
    $$
\end{thm}

\fi

\end{document}
